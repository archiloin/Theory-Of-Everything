\documentclass[12pt]{article}
\usepackage[utf8]{inputenc}
\usepackage[T1]{fontenc}
\usepackage{amsmath,amssymb,amsfonts,amsthm}
\usepackage{hyperref}
\usepackage{lmodern}
\usepackage{geometry}
\usepackage[french]{babel}
\usepackage{csquotes}
\geometry{margin=1in}

\begin{document}

\title{\textbf{Analyse D\'etaill\'ee de l'Action Unifi\'ee :}\\
\textbf{\emph{Nous ne sommes qu'un}}}
\author{Projet AIO (Alpha to Omega)}
\date{\today}
\maketitle

\begin{abstract}
Nous pr\'esentons ici un examen d\'etaill\'e d'une \emph{action unifi\'ee} comprenant 
gravitation \`a la Einstein-Hilbert, champs de jauge (Yang-Mills), mati\`ere fermionique 
et champs scalaires responsables de la brisure de sym\'etrie. 
Chaque terme est pass\'e en revue, depuis la mesure d'int\'egration \(\sqrt{-g}\) 
jusqu'aux corrections et termes suppl\'ementaires (SUSY, invariants topologiques, etc.). 
Nous montrons comment cette formulation illustre l'id\'ee que, \`a haute \'energie, 
toutes les forces et toute la mati\`ere se r\'eunissent en un \emph{m\^eme} cadre, 
appuyant la philosophie \og Nous ne sommes qu'un\fg.
\end{abstract}

\tableofcontents

\vspace{2em}

\section{Introduction}

Dans le cadre du projet AIO (\emph{Alpha to Omega}), nous proposons de d\'ecrire 
la \textbf{th\'eorie unifi\'ee} via une \emph{action unique}~:
\begin{equation}
\label{eq:U_action}
U 
\;=\;
\int d^4x \,\sqrt{-g}\;
\Bigl[
\frac{1}{2\kappa^2}\,R(g)
\;-\;\Lambda
\;-\;\tfrac{1}{4}\,F_{\mu\nu}^A\,F^{\mu\nu A}
\;+\;\overline{\Psi}\,\bigl(i\,\gamma^\mu D_\mu\bigr)\,\Psi
\;+\;\lvert D_\mu \Phi\rvert^2
\;-\;V(\Phi)
\;+\;\Delta_{\text{Yukawa}}
\;+\;\cdots
\Bigr]
\;+\;
S_{\text{corrections}}.
\end{equation}
Cette action se veut un \emph{cadre global} incluant la relativit\'e g\'en\'erale, 
les champs de jauge, la mati\`ere fermionique et le champ scalaire (Higgs ou plus g\'en\'eral) 
responsable de la brisure de sym\'etrie. L'objectif est de souligner comment, \`a haute \'energie, 
ces divers secteurs \og se confondent\fg{} en une \emph{m\^eme structure} --- r\'esumant 
l'id\'ee \og Nous ne sommes qu'un\fg.

Dans les sections suivantes, nous d\'etaillons \textbf{termes par termes} 
les diff\'erents blocs de l'action, avant de d\'emontrer comment la variation 
(\(\delta U = 0\)) g\'en\`ere les lois habituelles (\'equations d'Einstein, de Yang-Mills, etc.) 
et comment l'\'evolution cosmique entra\^ine la \textbf{d\'esunification} apparente 
des forces \`a basse \'energie.

\section{Termes de l'Action : Analyse d\'etaill\'ee}

\subsection{(1) La mesure d’int\'egration : \(\displaystyle \int d^4 x \,\sqrt{-g}\)}

\begin{itemize}
\item \(\displaystyle d^4x\) : symbolise l'int\'egration sur les \textbf{4 dimensions} 
      de l'espace-temps (ex. \((t,x,y,z)\)). 
\item \(\sqrt{-g}\) : 
   \begin{itemize}
   \item \(g = \det(g_{\mu\nu})\), la signature est souvent \((-,+,+,+)\) (d'o\`u le signe).
   \item Refl\`ete la \textbf{m\'etrique} courbe, assurant l'\textbf{invariance covariante} 
         sous changements de coordonn\'ees.
   \end{itemize}
\end{itemize}
\paragraph{R\^ole global :} 
Permet de \og peser \fg{} correctement la densit\'e lagrangienne dans un espace-temps 
relativiste courbe, \emph{invariant} sous diff\'eomorphismes.

\subsection{(2) \(\displaystyle \frac{1}{2\,\kappa^2}\,R(g)\) : le terme gravitationnel}

\begin{itemize}
\item \(R(g)\) : le \textbf{scalaire de courbure} calcul\'e via 
      le tenseur de Ricci \(R_{\mu\nu}\).
\item \(\kappa^2 = 8\pi G\) : param\`etre reliant la physique de la courbure \`a la constante de Newton \(G\).
\item Action d'Einstein-Hilbert : \(\tfrac{1}{2\kappa^2}\int d^4x \,\sqrt{-g}\,R\).
\end{itemize}
\paragraph{R\^ole global :}
\textbf{Dynamique de la gravit\'e}~; en posant \(\delta g_{\mu\nu}=0\), 
on obtient les \textbf{\'equations d'Einstein} reliant la g\'eom\'etrie de l'espace-temps 
\`a la distribution d'\'energie-impulsion.

\subsection{(3) \(\displaystyle -\Lambda\) : la constante cosmologique}

\begin{itemize}
\item \(\Lambda\) : terme d'\'energie du vide (au sens cosmologique).
\item Contribue aux \textbf{\'equations d'Einstein} sous forme d'un fluide homog\`ene.
\end{itemize}
\paragraph{R\^ole global :}
Explique \textbf{l'expansion} ou l'\textbf{acc\'el\'eration} cosmique 
(mod\`ele \(\Lambda\mathrm{CDM}\) actuel).

\subsection{(4) \(\displaystyle -\tfrac{1}{4}\,F_{\mu\nu}^A\,F^{\mu\nu A}\) : champ de jauge}

\begin{itemize}
\item \(F_{\mu\nu}^A\) : tenseur de champ \emph{Yang-Mills} pour un groupe de jauge (ex. \(\mathrm{SU}(N)\)).
\item \(\displaystyle F_{\mu\nu}^A 
        = \partial_\mu A_\nu^A - \partial_\nu A_\mu^A + g\,f^{ABC}A_\mu^B\,A_\nu^C\).
\item Repr\'esente la \textbf{dynamique} des bosons de jauge (forte, \'electrofaible, etc.).
\end{itemize}
\paragraph{R\^ole global :}
D\'efinit les \textbf{forces de jauge}, unifi\'ees \`a haute \'energie dans un groupe \(\mathcal{G}\).

\subsection{(5) \(\displaystyle \overline{\Psi}\,\bigl(i\,\gamma^\mu D_\mu\bigr)\,\Psi\) : la mati\`ere (fermions)}

\begin{itemize}
\item \(\Psi\) : champ de \textbf{fermions} (quarks, leptons).
\item \(\gamma^\mu\) : matrices de Dirac, \(\{ \gamma^\mu,\gamma^\nu \}=2g^{\mu\nu}\).
\item \(D_\mu\) : \textbf{d\'eriv\'ee covariante} incorporant le couplage aux bosons de jauge.
\end{itemize}
\paragraph{R\^ole global :}
D\'ecrit la \textbf{cin\'ematique} quantique relativiste et les interactions 
entre la mati\`ere et les champs de jauge.

\subsection{(6) \(\displaystyle |D_\mu \Phi|^2\) : terme cin\'etique du champ scalaire (Higgs)}

\begin{itemize}
\item \(\Phi\) : champ \textbf{scalaire} (ex. doublet de Higgs dans le MS ou multiplet plus grand en GUT).
\item \(|D_\mu \Phi|^2\) : \textbf{d\'eriv\'ee covariante} = couplage \`a la jauge, 
      d\'efinit la cin\'ematique du scalaire.
\end{itemize}
\paragraph{R\^ole global :}
Base de la \textbf{brisure de sym\'etrie} via un potentiel \(\,V(\Phi)\). 
Un VEV de \(\Phi\) diff\'erencie la force, conf\`ere la masse aux bosons.

\subsection{(7) \(\displaystyle -\,V(\Phi)\) : potentiel du scalaire (brisure de sym\'etrie)}

\begin{itemize}
\item Forme usuelle : \(V(\Phi) = \mu^2|\Phi|^2 + \lambda|\Phi|^4 + \cdots\).
\item \(\langle \Phi\rangle \neq 0\) \(\Rightarrow\) \textbf{brisure spontan\'ee de sym\'etrie}.
\end{itemize}
\paragraph{R\^ole global :}
\textbf{Oriente} la sym\'etrie \og unifi\'ee\fg{} vers des sous-groupes \`a basse \'energie, 
expliquant la \og d\'esunification\fg{} (forces distinctes).

\subsection{(8) \(\displaystyle \Delta_{\text{Yukawa}}\) : couplages fermions--Higgs}

\begin{itemize}
\item Terme typique : \(-\,y_{ij}\,\overline{\Psi_i}\,\Phi\,\Psi_j + \text{h.c.}\).
\item G\'en\`ere des \textbf{masses fermioniques} (quarks, leptons) apr\`es la brisure de sym\'etrie.
\end{itemize}
\paragraph{R\^ole global :}
Explique la \textbf{hi\'erarchie} des masses et le couplage Higgs--fermions dans le MS et GUT.

\subsection{(9) \(\displaystyle \cdots\) et \(\displaystyle S_{\text{corrections}}\)}

R\'ef\'erent \`a divers \textbf{termes suppl\'ementaires}:
\begin{itemize}
\item Supersym\'etrie, invariants topologiques (\(\theta\)-termes), 
      dimensions suppl\'ementaires (th\'eorie des cordes), etc.
\item \(\,S_{\text{corrections}}\) peut inclure l'inflaton (inflation cosmique), 
      la mati\`ere noire (BSM), etc.
\end{itemize}
\paragraph{R\^ole global :}
Affinement du mod\`ele de base. 
Peut \^etre crucial \`a \textbf{tr\`es haute \'energie} (ex. GUT, supergravité, etc.).

\section{Variation de l'Action : \(\,\delta U = 0\)}

En appliquant le \textbf{principe de moindre action}, on obtient :
\begin{itemize}
\item \(\delta g_{\mu\nu}\) $\to$ \textbf{\'equations d'Einstein} (Relativit\'e G\'en\'erale).
\item \(\delta A_\mu^A\) $\to$ \textbf{\'equations de Yang-Mills} (forces de jauge).
\item \(\delta \Psi\) $\to$ \textbf{\'equations de Dirac} (fermions).
\item \(\delta \Phi\) $\to$ \textbf{\'equation de Higgs}, brisure de sym\'etrie.
\end{itemize}
Cette unification formelle indique qu'\textbf{une seule expression} (l'action \eqref{eq:U_action}) 
suffit \`a g\'en\'erer la \emph{totalit\'e} des lois dynamiques des champs.

\section{Du Stade Unifi\'e \`a la D\'esunification :}

\subsection{(11.1) \`A haute \'energie}

\begin{itemize}
\item \(\Phi\) n'a pas de \textbf{valeur dans le vide} (VEV) : la sym\'etrie \(\mathcal{G}\) 
      est intacte (ex. GUT).
\item Gravitation potentiellement \emph{quantique}, possiblement \og unifi\'ee\fg{}
      \`a la m\^eme \'echelle.
\end{itemize}

\subsection{(11.2) Refroidissement cosmique}

\begin{itemize}
\item \(\,T\) diminue~; \(\Phi\) acquiert un VEV \`a une \'energie critique (ex. GUT).
\item \(\,\mathcal{G}\) se \og casse \fg{} en sous-groupes, 
      exposant la force forte, la force \'electrofaible, etc.
\end{itemize}

\subsection{(11.3) \'Etat actuel}

\begin{itemize}
\item On observe \(\mathrm{SU}(3)_\text{c}\times \mathrm{SU}(2)_L\times\mathrm{U}(1)_Y\) 
      \(\to\mathrm{U}(1)_{\mathrm{EM}}\), plus la gravit\'e classique.
\item \(\Lambda\neq0\) expliquerait l'acc\'el\'eration 
      \(\Rightarrow\) sc\'enario \(\Lambda\mathrm{CDM}\).
\end{itemize}

\section{Conclusion~: Pourquoi cette action illustre \og Nous ne sommes qu'un\fg}

\subsection{Unity \`a haute \'energie}

La m\^eme int\'egrale \eqref{eq:U_action} \textbf{englobe} la gravit\'e et 
les champs de jauge/mati\`ere, sugg\'erant qu'\`a \textbf{tr\`es haute \'energie}, 
ils forment une \emph{seule entit\'e coh\'erente}.

\subsection{Diff\'erenciation \`a basse \'energie}

La \textbf{brisure de sym\'etrie} via \(V(\Phi)\) cr\'ee l'\og illusion\fg{} 
de \textbf{multiplicit\'e} (plusieurs forces, vari\'et\'e de masses). 
Pourtant, \emph{toutes} \enquote{descendent} d'un \emph{m\^eme} sch\'ema.

\subsection{\'Evolution cosmique unifiante}

Le refroidissement depuis le \emph{Big Bang} agit comme un \og programmateur\fg{}
des diff\'erentes phases (GUT, \'electrofaible), 
expliquant comment la \textbf{d\'esunification} s'est produite 
mais laissant ouverte la possibilit\'e qu'\`a \(\sim 10^{15\text{-}19}\,\mathrm{GeV}\), 
\og Nous ne sommes qu'un\fg.

\bigskip

\noindent
\textbf{R\'ecapitulatif :} \\
L'action \(\,U\) rassemble :
\begin{itemize}
\item \(\frac{1}{2\kappa^2}\,R - \Lambda\) : \textbf{gravitation}, 
\item \(-\tfrac14\,F_{\mu\nu}^A\,F^{\mu\nu A}\) : \textbf{champs de force (Yang-Mills)},
\item \(\overline{\Psi}\,(i\gamma^\mu D_\mu)\,\Psi\) : \textbf{fermions},
\item \(|D_\mu \Phi|^2 - V(\Phi)\) : \textbf{Higgs} (brisure de sym\'etrie),
\item \(\Delta_{\mathrm{Yukawa}}\) : \textbf{couplages de masse},
\item \(\cdots\;+\;S_{\text{corrections}}\) : \textbf{extensions} (SUSY, cordes, etc.).
\end{itemize}

En appliquant \(\delta U=0\), on obtient l'ensemble des \emph{\'equations de la physique} 
(gravitation, interactions de jauge, masses, etc.). \`A \textbf{haute \'energie}, 
le cadre se referme en un tout coh\'erent (ex. grande unification, supergravité), 
d'o\`u la conclusion~: 
\[
\boxed{\text{\it ``Nous ne sommes qu'un'' : les quatre forces et la mati\`ere partagent une m\^eme origine unifi\'ee.}}
\]

\vspace{1em}

\begin{thebibliography}{9}

\bibitem{weinberg1995quantum} 
S. Weinberg, 
\textit{The Quantum Theory of Fields}, 
Cambridge University Press, 1995.

\bibitem{zee2010qft}
A. Zee,
\textit{Quantum Field Theory in a Nutshell},
Princeton University Press, 2010.

\bibitem{rovelli2004quantum} 
C. Rovelli, 
\textit{Quantum Gravity}, 
Cambridge University Press, 2004.

\bibitem{langacker1981grand} 
P. Langacker,
``Grand Unified Theories and Proton Decay,''
\textit{Phys. Rept.}, 72, 185--385, 1981.

\bibitem{georgi1974unified} 
H. Georgi \& S. L. Glashow,
``Unity of All Elementary-Particle Forces,''
\textit{Phys. Rev. Lett.}, 32, 438--441, 1974.

\bibitem{amaldi1991precision} 
U. Amaldi, W. de Boer, \& H. F\"urstenau,
``Comparison of grand unified theories with electroweak and strong coupling constants measured at LEP,''
\textit{Phys. Lett. B}, 260(3-4), 447--455, 1991.

\end{thebibliography}

\end{document}
