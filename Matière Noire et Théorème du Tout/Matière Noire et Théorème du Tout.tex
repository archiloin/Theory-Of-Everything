\documentclass[12pt]{article}
\usepackage[utf8]{inputenc}
\usepackage[T1]{fontenc}
\usepackage{lmodern}
\usepackage{amsmath,amssymb,amsfonts}
\usepackage{hyperref}
\usepackage{graphicx}
\usepackage{geometry}
\usepackage{csquotes}
\usepackage[french]{babel}
\pdfstringdefDisableCommands{%
  \def\Phi{Phi}%
  \def\Psi{Psi}%
  \def\kappa{kappa}%
  \def\gamma{gamma}%
  \def\delta{delta}%
  \def\mu{mu}%
  \def\nu{nu}%
  \def\lambda{lambda}%
  \def\alpha{alpha}%
  \def\beta{beta}%
  \def\leq{<=}%
  \def\geq{>=}%
  \def\int{\string\int}%
}
\sloppy
\overfullrule=5pt
\geometry{margin=1in}

\title{\textbf{Matière Noire et Théorème du Tout :}\\
Comment la composante sombre s'intègre dans la vision unifiée \\
\og Nous ne sommes qu’un \fg}
\author{Projet AIO (Alpha to Omega)}
\date{\today}

\begin{document}

\maketitle

\begin{abstract}
La matière noire représente environ 85\% de la matière de l'Univers, tout en étant quasi invisible électromagnétiquement. Cet exposé montre comment, dans le cadre du \emph{Théorème du Tout} (\og Nous ne sommes qu’un \fg) proposant une unification des forces et de la matière à haute énergie, la matière noire peut naturellement émerger via de nouvelles particules ou de nouveaux secteurs de symétrie. Après un rappel sur les principales caractéristiques de la matière noire et sa nécessité cosmologique, nous expliquons pourquoi les scénarios de grande unification ou de physique BSM (Beyond the Standard Model) offrent fréquemment des candidats sombres stables, en cohérence avec l’idée qu’à haute énergie, toute la matière (claire ou sombre) provient d’une même racine unique. Enfin, nous discutons de la chronologie cosmique de la matière noire et des efforts expérimentaux pour la détecter, concluant qu’elle est un pan incontournable de la vision \og Nous ne sommes qu’un \fg.
\end{abstract}

\tableofcontents

\section{Rappel : qu’est-ce que la matière noire\,?}
\label{sec:rappel}

\subsection{Observations astrophysiques et cosmologiques}

Depuis plusieurs décennies, diverses observations indiquent qu’une grande partie de la masse de l’Univers n’interagit pas ou très peu avec le rayonnement électromagnétique : on parle de \textbf{matière noire} (\emph{dark matter}) \cite{zwicky1933rotational, rubin1980rotational, planck2018}. Parmi les indices majeurs :

\begin{itemize}
  \item \textbf{Courbes de rotation des galaxies}~: elles restent plates à grande distance, contrairement à ce que prédirait la seule matière baryonique.
  \item \textbf{Lentilles gravitationnelles}~: la déviation de la lumière par des amas de galaxies témoigne d’une masse \og invisible \fg.
  \item \textbf{Cosmologie}~: le modèle standard de la cosmologie \(\Lambda \mathrm{CDM}\) suggère qu’environ 85\,\% de la matière est \og noire \fg, non baryonique.
\end{itemize}

\subsection{Limitations du Modèle Standard}

Dans le Modèle Standard (MS) \cite{weinberg1995quantum, zee2010qft}, \emph{aucune} particule connue ne peut constituer à elle seule la totalité de la matière noire :
\begin{itemize}
  \item Les neutrinos, certes neutres et légers, sont \og trop chauds \fg{} et ne peuvent former toute la composante sombre.
  \item Les baryons, protons et neutrons, ne correspondent pas aux abondances requises (seulement \(\sim 15\,\%\) de la masse totale en tant que matière baryonique).
\end{itemize}
Ainsi, la matière noire \emph{nécessite} une \textbf{extension BSM (Beyond the Standard Model)} : nouvelles particules, nouvelles symétries, etc.

\subsection{Propriétés clés de la matière noire}

\begin{itemize}
    \item \textbf{Non baryonique} : elle n’est pas formée de protons ou neutrons ordinaires.
    \item \textbf{Stable (ou quasi stable)} : elle doit subsister depuis l’Univers primordial jusqu’à nos jours.
    \item \textbf{Faible interaction} avec le secteur visible : elle n’émet pas de lumière et n’interagit que via la gravité (et éventuellement d’autres interactions faibles ou cachées).
\end{itemize}

\section{Intégrer la matière noire dans le \og Théorème du Tout \fg}
\label{sec:theoreme_mn}

\subsection{Rappel du Théorème \og Nous ne sommes qu’un \fg}

Le \emph{Théorème du Tout} postule qu’à \textbf{très haute énergie}, toutes les interactions (gravitation, forces de jauge) et la matière (fermions, bosons) s’unifient ou tendent à former un cadre unique \cite{langacker1981grand, georgi1974unified}. La \emph{brisure de symétrie} causée par le refroidissement cosmique engendre la séparation apparente en quatre forces, ainsi que la distinction entre divers types de particules.

\begin{quote}
\emph{«\,Nous ne sommes qu’un\,» signifie que toute la matière et toutes les interactions -- qu’elles soient dites \og ordinaires \fg{} ou \og sombres \fg{} -- proviennent d’une même racine unifiée.}
\end{quote}

\subsection{Nouvelles particules stables dans les théories unifiées}

Dans de nombreux scénarios BSM (souvent liés à la \textbf{grande unification}, la supersymétrie, etc.), l’extension du groupe de jauge ou l’apparition de symétries supplémentaires prédit des \emph{nouvelles particules} qui :

\begin{itemize}
  \item Sont \textbf{électriquement neutres}, donc invisibles en électromagnétique.
  \item Pourraient être \textbf{stables} grâce à des symétries discrètes (ex. R-parité en supersymétrie).
  \item Ont une \textbf{interaction faible} voire quasi nulle avec le Modèle Standard.
\end{itemize}

\textbf{Exemples courants} \cite{jungman1996supersymmetric, feng2010dark} :
\begin{enumerate}
  \item \textbf{WIMPs (Weakly Interacting Massive Particles)} : neutralino, gravitino, etc., issus de la supersymétrie.
  \item \textbf{Axions} : particules légères hypothétiques (problème CP fort), parfois associées à un secteur \(\mathrm{SU}(3)\) étendu.
  \item \textbf{Particules exotiques GUT} : représentations spécifiques (fermions lourds, bosons singuliers) dans \(\mathrm{SU}(5)\), \(\mathrm{SO}(10)\), pouvant rester stables et invisibles.
\end{enumerate}

\subsection{Brisure de symétrie et relic abundance}

Lors du \textbf{refroidissement} cosmique, ces nouvelles particules sombres peuvent se \textbf{découpler} thermiquement du plasma primordial :

\begin{itemize}
  \item Elles cessent de s’annihiler suffisamment tôt si leur interaction est \emph{faible}, restant en nombre \emph{non négligeable} (\textit{freeze-out}).
  \item Elles constituent alors une \emph{densité relique} qui, aujourd’hui, représente la part dominante de la matière.
\end{itemize}

Ainsi, la matière noire \emph{émerge} naturellement dans les scénarios unifiés : le \og secteur sombre \fg{} est simplement un \emph{bout de la symétrie} qui \emph{ne} se couple \emph{pas} ou très peu à la force électromagnétique.

\subsection{Cohérence avec l’action unifiée \(\,U\)}

Dans l’action unifiée \(\,U\) (Einstein-Hilbert + Yang-Mills + champs scalaires + fermions) \cite{weinberg1995quantum, zee2010qft} :
\[
U
\;=\;
\int d^4x\,\sqrt{-g}\;\Bigl[\,
\tfrac{1}{2\kappa^2}\,R
-\tfrac14\,F_{\mu\nu}^A F^{\mu\nu A}
+\overline{\Psi}(i\gamma^\mu D_\mu)\Psi
+|D_\mu \Phi|^2 - V(\Phi)
+\ldots
\Bigr].
\]
il suffit de \textbf{rajouter} :
\begin{itemize}
  \item Des \emph{champs supplémentaires} (fermions, bosons) \emph{sans charge électrique}.
  \item Des couplages \emph{non nuls} mais \emph{faibles} entre ces champs et la matière ordinaire.
  \item Une \emph{symétrie de conservation} (p.~ex. R-parité) qui les rend stables.
\end{itemize}
La brisure de symétrie principale (GUT ou électrofaible) peut alors \emph{séparer} ces champs du secteur visible (\textit{hidden sector}), restant \og cachés \fg{} à basse énergie.

\section{Pourquoi la matière noire s’intègre naturellement au \og Nous ne sommes qu’un \fg}
\label{sec:integration}

\begin{enumerate}
  \item \textbf{Même origine fondamentale} : matière claire ou sombre, tout provient d’un \emph{même cadre unifié}. Les \og propriétés sombres \fg{} découlent simplement d’un couplage très faible au secteur électromagnétique.
  \item \textbf{Nécessité de BSM} : l’existence de la matière noire confirme que le Modèle Standard \emph{n’est pas complet}, cohérent avec la \emph{grande unification} et d’autres extensions.
  \item \textbf{Désunification} : la brisure de symétrie explique pourquoi certains champs acquièrent \og une visibilité \fg{} (force électromagnétique) alors que d’autres demeurent \og invisibles \fg{} (secteur sombre).
\end{enumerate}

\section{Chronologie cosmique de la matière noire}
\label{sec:chronologie_mn}

\begin{itemize}
  \item \textbf{Avant ou pendant la brisure électrofaible} : si la matière noire a une masse typique de l’ordre \(100\text{-}10^3\,\mathrm{GeV}\), elle se \emph{gèle} (\emph{freeze-out}) lors de la période où l’Univers atteint ces températures.
  \item \textbf{Après la brisure GUT} : si elle découle d’un secteur caché, elle peut être \og isolée \fg{} de notre secteur standard très tôt, alors que la force forte et l’électrofaible ne sont pas encore séparées.
  \item \textbf{Aujourd’hui} : les particules sombres forment des \emph{halos} autour des galaxies et influencent la structure à grande échelle de l’Univers.
\end{itemize}

\section{Point de vue expérimental}
\label{sec:experiment}

Plusieurs \textbf{approches} visent à détecter la matière noire \cite{darkmatterreview} :

\begin{enumerate}
  \item \textbf{Détection directe} : détecteurs souterrains (XENON, LZ, etc.) enregistrant un \emph{recul nucléaire} induit par un WIMP.
  \item \textbf{Détection indirecte} : recherche de signaux d’annihilation (photons, positrons, neutrinos) émis par des particules sombres dans la Galaxie.
  \item \textbf{Accélérateurs} : productions de particules invisibles au LHC (événements avec \og missing energy \fg).
  \item \textbf{Observations astrophysiques/cosmologiques} : analyse fine des \emph{lentilles gravitationnelles}, du \textbf{CMB}, et de la distribution de la matière à grande échelle.
\end{enumerate}

La \emph{découverte} d’une nouvelle particule stable, neutre, et jouant le rôle de matière noire, \emph{confirmerait} directement qu’une physique \emph{au-delà du MS} existe.

\section{Conclusion : la matière noire, un chapitre indispensable du \og Nous ne sommes qu’un \fg}
\label{sec:conclusion}

\noindent
Dans le \textbf{cadre du Théorème «\,Nous ne sommes qu’un\,»}, l’unification des forces à haute énergie implique naturellement qu’il existe \emph{plus} de degrés de liberté que ceux du Modèle Standard seul. La présence d’un \og secteur sombre \fg{}, faiblement couplé et responsable de la matière noire, s’insère harmonieusement :

\begin{itemize}
    \item \emph{La matière noire} n’est pas une \og anomalie \fg, mais le signe que la \emph{dynamique d’unification} cache un \emph{bloc de particules} invisibles à basse énergie.
    \item \emph{L’unité} se reflète dans le fait que ces champs sombres et les champs visibles \emph{partagent} la même structure fondamentale avant la \emph{brisure de symétrie}.
    \item \emph{La brisure} (GUT, électrofaible, etc.) \emph{sépare} progressivement la matière claire et la matière sombre, conduisant à la \og pluralité \fg{} actuelle.
\end{itemize}

Ainsi, la \textbf{matière noire} est \emph{un élément clé} du puzzle unificateur : elle \og réaffirme \fg{} que la physique standard ne constitue qu’un fragment d’une \emph{réalité plus vaste}, contenant une ou plusieurs composantes invisibles. Dans cette perspective, \emph{clair ou sombre, tout se rattache à la même origine}, validant la \textbf{philosophie} \og Nous ne sommes qu’un \fg.

\vspace{0.5cm}

\begin{thebibliography}{9}

\bibitem{zwicky1933rotational}
F. Zwicky, 
``Die Rotverschiebung von extragalaktischen Nebeln,''
\textit{Helv. Phys. Acta}, 6, 110--127, 1933.

\bibitem{rubin1980rotational}
V. C. Rubin,
``Rotational properties of 21 SC galaxies with a large range of luminosities and radii, from NGC 4605 / R = 4kpc / to UGC 2885 / R = 122kpc /,''
\textit{Astrophys. J.}, 238, 471--487, 1980.

\bibitem{planck2018}
Planck Collaboration,
\textit{Planck 2018 results. VI. Cosmological parameters},
\textit{Astronomy \& Astrophysics}, 641, A6, 2020.

\bibitem{weinberg1995quantum} 
S. Weinberg,
\textit{The Quantum Theory of Fields},
Cambridge University Press, 1995.

\bibitem{zee2010qft}
A. Zee,
\textit{Quantum Field Theory in a Nutshell},
Princeton University Press, 2010.

\bibitem{langacker1981grand} 
P. Langacker,
``Grand Unified Theories and Proton Decay,''
\textit{Phys. Rept.}, 72, 185--385, 1981.

\bibitem{georgi1974unified}
H. Georgi and S. L. Glashow,
``Unity of All Elementary-Particle Forces,''
\textit{Phys. Rev. Lett.}, 32, 438--441, 1974.

\bibitem{jungman1996supersymmetric}
G. Jungman, M. Kamionkowski, and K. Griest,
``Supersymmetric dark matter,''
\textit{Phys. Rep.}, 267, 195--373, 1996.

\bibitem{feng2010dark}
J. L. Feng,
``Dark Matter Candidates from Particle Physics and Methods of Detection,''
\textit{Ann. Rev. Astron. Astrophys.}, 48, 495--545, 2010.

\bibitem{darkmatterreview}
G. Bertone, D. Hooper, and J. Silk,
``Particle dark matter: evidence, candidates and constraints,''
\textit{Phys. Rept.}, 405, 279--390, 2005.

\end{thebibliography}

\end{document}
