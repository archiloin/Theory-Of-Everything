%=======================================================================================
% Fichier : annB.tex
% Annexe B : Lemmes Techniques de Renormalisation
%=======================================================================================
\chapter{Lemmes Techniques de Renormalisation}
\label{ann:B}

%---------------------------------------------------------------------------------------
% Section B.1 : Le “RG Flow” de Balaban : énoncé rigoureux
%---------------------------------------------------------------------------------------
\section{Le “RG Flow” de Balaban : Énoncé Rigoureux}
\label{sec:B.1}

\subsection*{Présentation du flux RG}
Le \textbf{groupe de renormalisation} (RG) décrit la transformation d’une théorie lorsqu’on \emph{intègre} les fluctuations d’échelle supérieure à \(\Lambda_{k+1}\) pour passer à \(\Lambda_k\). T.~Balaban \cite{Balaban1982-1,Balaban1982-2} a formulé ces intégrations partielles de manière \textbf{non perturbative}.

\subsection*{Énoncé type}
\begin{theorem}[Balaban’s RG Flow]
	Il existe une \textbf{suite de transformations} \(\{T_k\}\) sur l’espace fonctionnel des actions gauge-invariantes telle que, pour chaque échelle \(k\), la nouvelle action \(\widetilde{S}_k\) demeure dans la \emph{même classe} de fonctions gauge-invariantes et vérifie les \textbf{estimations} de stabilité :
	\[
	\|\widetilde{S}_k - S_{k}\| \;\le\; C\,\alpha_k,
	\]
	où \(\alpha_k\to 0\) pour \(\,k\to \infty\), assurant la \textbf{convergence} vers une \emph{théorie limite} (Yang--Mills 4D).
\end{theorem}

\subsection*{Idée de la preuve}
La preuve utilise des décompositions en blocs, le contrôle des \og graphes d’arbres\fg\ et une \textbf{fixation de jauge partielle} pour éliminer les redondances. L’important est de conserver la \emph{structure non abélienne} intacte, ce qui n’est pas trivial.

\vspace{1em}

%---------------------------------------------------------------------------------------
% Section B.2 : Estimations d’Intégrales : expansions de Mayer/cluster
%---------------------------------------------------------------------------------------
\section{Estimations d’Intégrales : Expansions de Mayer/Cluster}
\label{sec:B.2}

\subsection*{Expansions en cluster}
Dans les approches constructives, on emploie souvent les \textbf{expansions de Mayer} (ou \textit{polymer expansions}) pour réorganiser la somme/inégrale en ensembles \(\{\Gamma\}\) de \og clusters\fg\ :
\[
Z \;=\; \exp\Bigl(\sum_{\Gamma}\,\phi(\Gamma)\Bigr),
\]
où \(\phi(\Gamma)\) sont des contributions locales, décroissant rapidement avec la taille de \(\Gamma\).

\subsection*{But : majorer la somme}
On veut prouver que la somme (ou exponentielle) ne diverge pas, et qu’elle est contrôlée par un \textbf{facteur} exponentiel. Cela justifie la \textbf{décroissance exponentielle des corrélations} à grande distance et l’absence de \og pôle de masse nulle\fg.

\vspace{1em}

%---------------------------------------------------------------------------------------
% Section B.3 : Exemples Illustratifs (fictifs ou simplifiés)
%---------------------------------------------------------------------------------------
\section{Exemples Illustratifs (Fictifs ou Simplifiés)}
\label{sec:B.3}

\subsection*{Modèle \(\phi^4\) abélien}
Bien que différent de \(\mathrm{SU}(N)\), le modèle \(\phi^4\) (champ scalaire) en 4D sert de terrain d’entraînement. On sait qu’il admet aussi des expansions de Mayer, prouvant la renormalisabilité (Glimm--Jaffe \cite{GlimmJaffe1987}).

\subsection*{Version \(\mathrm{U}(1)\) lattice}
Pour \(\mathrm{U}(1)\) (électromagnétisme compact), on peut écrire des \textbf{chemins fermés} sur le réseau. L’extension à \(\mathrm{SU}(N)\) introduit de la non commutativité, mais le principe reste identique : reconstituer l’intégrale en \og blocs\fg\ ou \og liens\fg\ corrélés.

\vspace{2em}

%---------------------------------------------------------------------------------------
% Conclusion de l’Annexe B
%---------------------------------------------------------------------------------------
\noindent
\textbf{Conclusion de l’Annexe B :}\\
Les \textbf{lemmes techniques} de Balaban, Freedman, Rivasseau, etc. s’articulent autour de \emph{transformations d’échelle} rigoureuses (theorems RG) et d’\emph{estimations en cluster} contrôlant les divergences. C’est le \textbf{noyau dur} de la \emph{preuve constructive} du Mass Gap en 4D.  

\vspace{2em}

%=======================================================================================
% Références Bibliographiques (en dur) pour l’Annexe B
%=======================================================================================
\begin{thebibliography}{99}
	
	\bibitem{Balaban1982-1}
	T.~Balaban,
	\textit{Renormalization Group Approach to Lattice Gauge Field Theories (I)},
	Commun.~Math.~Phys. \textbf{79}, 277--321 (1981).
	\\[-0.75em]
	
	\bibitem{Balaban1982-2}
	T.~Balaban,
	\textit{Renormalization Group Approach to Lattice Gauge Field Theories (II)},
	Commun.~Math.~Phys. \textbf{83}, 363--376 (1982).
	\\[-0.75em]
	
	\bibitem{GlimmJaffe1987}
	J.~Glimm, A.~Jaffe,
	\textit{Quantum Physics: A Functional Integral Point of View},
	2nd ed., Springer, New York (1987).
	
\end{thebibliography}

%=======================================================================================
% Fin du fichier : annB.tex
%=======================================================================================
