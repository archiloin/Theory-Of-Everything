%=======================================================================================
% Fichier : annC.tex
% Annexe C : Méthodes Numériques sur Réseau
%=======================================================================================
\chapter{Méthodes Numériques sur Réseau}
\label{ann:C}

%---------------------------------------------------------------------------------------
% Section C.1 : Algorithmes de Monte Carlo pour SU(N)
%---------------------------------------------------------------------------------------
\section{Algorithmes de Monte Carlo pour \texorpdfstring{\(\mathrm{SU}(N)\)}{SU(N)}}
\label{sec:C.1}

\subsection*{Principe de Metropolis/HMC}
Pour échantillonner les configurations \(\{U_\ell\}\) selon la mesure
\(\mathrm{d}\mu_{\mathrm{Haar}}(U_\ell)\,\exp(-\beta S_{\mathrm{W}})\),
on utilise des algorithmes Monte Carlo :
\begin{itemize}
	\item \textbf{Metropolis} : on propose une modification locale \(\delta U_\ell\), 
	puis on l’accepte ou on la rejette selon une probabilité proportionnelle 
	à \(\exp(-\Delta S)\).
	\item \textbf{HMC (Hybrid Monte Carlo)} : on introduit des \og moments\fg\ 
	(variables de moment) et on réalise une intégration \(\mathrm{d}H = \mathrm{d}p\,\mathrm{d}U\)
	suivie d’un accept/reject global.
\end{itemize}

\subsection*{Effets d’autocorrélation}
Afin d’obtenir des configurations suffisamment \emph{indépendantes}, il faut
laisser \textbf{plusieurs pas} entre deux mesures (pas MC). À couplage fort,
ce \og temps de relaxation\fg\ peut devenir important, phénomène connu
sous le nom de \textbf{critical slowing down}.

\vspace{1em}

%---------------------------------------------------------------------------------------
% Section C.2 : Évaluations Numériques du Gap : revue de résultats classiques
%---------------------------------------------------------------------------------------
\section{Évaluations Numériques du Gap : Revue de Résultats Classiques}
\label{sec:C.2}

\subsection*{\(\mathrm{SU}(2)\) et \(\mathrm{SU}(3)\)}
Historiquement, M.~Creutz \cite{Creutz1983} a réalisé des simulations
pionnières montrant la \textbf{croissance linéaire du potentiel} (confinement)
et l’existence d’un glueball massif. Ensuite, Teper \cite{Teper1998} affina
ces valeurs, trouvant, par exemple :
\[
m_{0^{++}} \approx 1.6 \,\mathrm{GeV} \quad \text{pour SU(3) pure,}
\]
confirmé par d’autres collaborations (UKQCD, etc.).

\subsection*{Grilles plus fines et supercalculateurs}
Depuis les années 2000, de grandes collaborations QCD (RBC-UKQCD, MILC, etc.)
emploient des grilles encore plus fines (\(32^4,\,64^4\), etc.) et affinent
les extrapolations \(\,a \to 0\). Toutes ces données confirment la
\textbf{convergence} vers un \emph{mass gap} non nul pour la théorie pure
\(\mathrm{SU}(N)\).

\vspace{1em}

%---------------------------------------------------------------------------------------
% Section C.3 : Comparaison Analytique/Numérique
%---------------------------------------------------------------------------------------
\section{Comparaison Analytique/Numérique}
\label{sec:C.3}

\subsection*{Complémentarité}
Les approches:
\begin{itemize}
	\item \textbf{Constructives / Renormalisation multi-échelle} : donnent une 
	preuve rigoureuse de l’existence du mass gap (\(\Delta>0\)), mais ne visent 
	pas nécessairement une valeur numérique précise.
	\item \textbf{Simulations Lattice} : permettent d’\emph{estimer} quantitativement
	le gap et de tester la théorie à divers paramètres \(\beta, L, a\).  
\end{itemize}

\subsection*{Vers un mariage analytique-numérique}
Le \textbf{rêve} de la communauté :  
\begin{enumerate}
	\item Des preuves analytiques robustes (contrôle complet des divergences et
	de l’invariance de jauge) démontrant la finitude et \(\Delta>0\).  
	\item Des \emph{simulations} haute précision pour affiner la valeur numérique
	du gap (ex. \(\Delta \approx 1.6\,\mathrm{GeV}\) pour \(\mathrm{SU}(3)\)).
\end{enumerate}
Ainsi, la QCD pure en 4D apparaît fondée sur un solide \emph{début} de preuve
constructive et confortée par l’évidence numérique.

\vspace{2em}

%---------------------------------------------------------------------------------------
% Conclusion de l’Annexe C
%---------------------------------------------------------------------------------------
\noindent
\textbf{Conclusion de l’Annexe C :}\\
Les méthodes numériques sur réseau constituent le \textbf{pilier empirique}
de l’étude du Mass Gap pour Yang--Mills. Elles \textbf{confirment} que le
confinement et la présence d’un gap \(\Delta \approx 1.6\,\mathrm{GeV}\) (dans
le cas \(\mathrm{SU}(3)\) pur) sont parfaitement plausibles. Couplées aux
\emph{techniques constructives}, elles offrent un panorama cohérent,
à la fois qualitatif (preuve formelle) et quantitatif (simulation).

\vspace{2em}

%=======================================================================================
% Références Bibliographiques (en dur) pour l’Annexe C
%=======================================================================================
\begin{thebibliography}{99}
	
	\bibitem{Creutz1983}
	M.~Creutz,
	\textit{Quarks, Gluons and Lattices},
	Cambridge University Press, Cambridge (1983).
	\\[-0.75em]
	
	\bibitem{Teper1998}
	M.~Teper,
	\textit{Glueball Masses and Other Physical Properties of SU(N) Gauge Theories in D=3+1: A Review of Lattice Results for Theorists},
	arXiv:hep-th/9812187 (1998).
	
\end{thebibliography}

%=======================================================================================
% Fin du fichier : annC.tex
%=======================================================================================
