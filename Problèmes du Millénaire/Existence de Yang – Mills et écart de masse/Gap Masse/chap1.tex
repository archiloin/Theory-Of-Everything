%=======================================================================================
% Fichier : chap1.tex
% Chapitre 1 : Présentation Générale du Problème
%=======================================================================================
\chapter{Présentation Générale du Problème}
\label{chap:1}

%---------------------------------------------------------------------------------------
% Section 1.1 : Le Mass Gap de Yang--Mills : Énoncé du Problème
%---------------------------------------------------------------------------------------
\section{Le Mass Gap de Yang--Mills : Énoncé du Problème}
\label{sec:1.1}

Le \emph{Mass Gap} dans une théorie de Yang--Mills (YM) en 4 dimensions désigne l’existence d’un \textbf{écart d’énergie strictement positif}, noté \(\Delta\), entre l’état fondamental (\og vacuum\fg) et la première excitation du spectre. Concrètement, si l’on reconstruit la théorie en temps réel et que l’on définit un hamiltonien quantique \(\widehat{H}\), cette première excitation correspond à une particule (ou \og glueball\fg) de masse \(\Delta\). Démontrer \(\Delta > 0\) revient donc à prouver que \textbf{tous} les excitations du champ de jauge ont une masse finie, aucune n’étant au niveau zéro.

\subsection*{Formulation succincte du problème}

\begin{itemize}
	\item \textbf{Définir} rigoureusement la mesure de la théorie de Yang--Mills en 4D (existence non-perturbative).
	\item \textbf{Montrer} que le spectre est discret et qu’il existe un \(\Delta > 0\).
	\item \textbf{Assurer} que la symétrie de jauge \(\mathrm{SU}(N)\) n’est pas brisée.
\end{itemize}

Pour avancer vers cette preuve, on utilise deux grandes familles d’approches : la \textbf{formulation sur réseau} (lattice) et les \textbf{méthodes constructives multi-échelles}. Celles-ci permettent de contrôler la théorie dans la limite \emph{ultraviolet} (\(a \to 0\)) tout en conservant l’invariance de jauge. Les grandes lignes de ces méthodes seront exposées en partie~II.

\begin{proof}[Preuves directes ou indirectes (aperçu)]
	\textbf{Preuves indirectes :}
	\begin{itemize}
		\item \textbf{Simulations sur réseau :} dès les années 1970, \cite{Wilson1974} a proposé la discrétisation (lattice) pour la QCD ; les calculs numériques ultérieurs (voir \cite{Creutz1980} entre autres) ont clairement montré l’absence d’excitations massives nulles, donc un \emph{gap} non nul.
		\item \textbf{Faits expérimentaux :} on n’a jamais observé de particule gluonique libre de masse nulle dans les collisions hadroniques, suggérant que les \og gluons\fg\ se retrouvent confinés dans des états massifs (glueballs).
	\end{itemize}
	
	\textbf{Preuves directes (objectif de ce manuscrit) :}
	\begin{itemize}
		\item \textbf{Construction mathématique} via des techniques de \emph{renormalisation constructive} : développement multi-échelles (Balaban \cite{Balaban1982-1,Balaban1982-2}, Rivasseau \cite{Rivasseau1991}) prouvant que la théorie en 4D est bien définie et finite dans la limite continuum.
		\item \textbf{Osterwalder--Schrader et reconstruction Minkowski} : la \textit{positivité réfléchie} permet de démontrer que le \og pôle de masse nulle\fg\ n’existe pas, d’où une décroissance exponentielle des corrélations, synonyme de mass gap.
	\end{itemize}
\end{proof}

%---------------------------------------------------------------------------------------
% Section 1.2 : Portée et Importance en Physique des Particules
%---------------------------------------------------------------------------------------
\section{Portée et Importance en Physique des Particules}
\label{sec:1.2}

\subsection*{Confinement et structure du Modèle Standard}
La théorie de Yang--Mills (typiquement \(\mathrm{SU}(3)\) pour la QCD) constitue le cœur de l’interaction forte dans le Modèle Standard. 
\begin{itemize}
	\item \textbf{Confinement} : si les excitations de couleur ne peuvent exister qu’à l’état lié, c’est en partie parce que le \emph{coût énergétique} pour séparer des charges colorées devient prohibitif à grande distance.  
	\item \textbf{Mécanisme de masse} : un \emph{écart} \(\Delta > 0\) est la signature mathématique que les gluons \og s’habillent\fg\ (se confinant entre eux), formant des \emph{glueballs} massifs.
\end{itemize}

Une théorie YM \(\mathrm{SU}(N)\) 4D avec \(\Delta > 0\) \textbf{explique} en grande partie la structure discrète des hadrons dans la nature, par opposition à d’éventuels champs de jauge \emph{libres} et dépourvus de masse.

\subsection*{Implications théoriques et calculatoires}
La \textbf{détermination du gap} a aussi des conséquences sur :
\begin{enumerate}
	\item \textbf{Perturbations autour du vide} : la théorie étant \emph{asymptotiquement libre} \cite{GrossWilczek1973,Politzer1973}, on sait qu’à haute énergie, le couplage devient petit ; mais la question est de savoir comment \emph{passer} au régime basse énergie de façon rigoureuse.  
	\item \textbf{Technologies numériques} : les résultats de Lattice QCD, validés par l’existence d’un gap, aident grandement à modéliser les observables hadroniques et à comparer avec l’expérience.  
\end{enumerate}

En bref, \textit{prouver le Mass Gap}, c’est assurer la cohérence mathématique de la QCD et fonder sur des bases plus \emph{solides} l’un des piliers du Modèle Standard.

%---------------------------------------------------------------------------------------
% Section 1.3 : Pourquoi est-ce un Problème du Millénaire ?
%---------------------------------------------------------------------------------------
\section{Pourquoi est-ce un Problème du Millénaire ?}
\label{sec:1.3}

Le Clay Mathematics Institute a placé \og Yang--Mills \& Mass Gap\fg\ parmi les sept \textbf{Problèmes du Millénaire}, témoignant de sa difficulté extrême et de sa portée fondamentale. 

\begin{itemize}
	\item \textbf{Existence rigoureuse d’une QFT 4D} : 
	Il ne suffit pas de \og définir formellement\fg\ une théorie par un lagrangien. Il faut prouver \emph{l’existence} d’une \textbf{mesure euclidienne} finie et invariante de jauge, vérifiant les axiomes Osterwalder--Schrader ou Wightman. 
	\item \textbf{Décroissance exponentielle des corrélations} :
	Une fois la théorie établie, il faut démontrer \(\langle O(x)\,O(y)\rangle \sim e^{-\Delta\,\|x-y\|}\) à grande distance, \emph{sans} pôle de masse nul. Cela nécessite d’utiliser des techniques de \textbf{positivité} (reflection positivity) et de renormalisation \emph{non abélienne}, deux domaines très techniques.
	\item \textbf{Divergences et multi-échelles} :
	En 4D, la renormalisation est un casse-tête subtil. S’assurer que \og rien ne diverge\fg\ (au sens pathologique) dans la limite \(\Lambda \to \infty\) ou \(a \to 0\) a requis des travaux de longue haleine, de Wilson \cite{Wilson1974} aux analyses sophistiquées de Balaban \cite{Balaban1982-1,Balaban1982-2}, Freedman, Rivasseau \cite{Rivasseau1991}, etc.
\end{itemize}

Ainsi, la communauté scientifique admet largement, sur la base de calculs numériques et d’arguments heuristiques, que \(\Delta > 0\). Mais la \textit{démonstration pleinement rigoureuse} reste, encore aujourd’hui, \emph{le} défi majeur que nous allons tenter de décortiquer.

%---------------------------------------------------------------------------------------
% Section 1.4 : Organisation Générale de la Démonstration
%---------------------------------------------------------------------------------------
\section{Organisation Générale de la Démonstration}
\label{sec:1.4}

\subsection*{Quatre grandes étapes}

Nous allons structurer cette démonstration en quatre parties, chacune correspondant à un \emph{bloc logique} :

\begin{enumerate}
	\item \textbf{Partie I : Introduction et Contexte} \\
	\underline{(Chapitres~\ref{chap:1} à \ref{chap:3})} \\
	Présentation du problème (chapitre~\ref{chap:1}, le présent chapitre), de l’historique et des motivations théoriques (chapitre~\ref{chap:2}), puis rappels de notations et préliminaires mathématiques (chapitre~\ref{chap:3}).
	
	\item \textbf{Partie II : Régularisations et Construction de la Théorie} \\
	\underline{(Chapitres~\ref{chap:4} à \ref{chap:6})} \\
	Introduction à la formulation sur réseau (chapitre~\ref{chap:4}) et aux méthodes constructives multi-échelles (chapitre~\ref{chap:5}). Nous comparerons et unifierons ces approches (chapitre~\ref{chap:6}) afin de préparer la limite continuum.
	
	\item \textbf{Partie III : Passage à la Limite Continuum \& Axiomes QFT} \\
	\underline{(Chapitres~\ref{chap:7} et \ref{chap:8})} \\
	Analyse du \og flot de renormalisation\fg\ dans la limite \(a \to 0\), preuve de la liberté asymptotique (chapitre~\ref{chap:7}), puis vérification des axiomes Osterwalder--Schrader et construction Minkowski (chapitre~\ref{chap:8}).
	
	\item \textbf{Partie IV : Preuve du Mass Gap et Conséquences Physiques} \\
	\underline{(Chapitres~\ref{chap:9} à \ref{chap:11})} \\
	Démonstration de la décroissance exponentielle des fonctions de corrélation (chapitre~\ref{chap:9}), existence d’un \(\Delta > 0\) (chapitre~\ref{chap:10}), et perspectives ouvertes (chapitre~\ref{chap:11}).
\end{enumerate}

\subsection*{Annexes et références}

Des annexes techniques sont fournies pour rappeler des outils mathématiques (annexe~A), formaliser certains lemmes de renormalisation (annexe~B) et présenter des méthodes numériques (annexe~C). Les références bibliographiques complètes, incluant les travaux fondateurs et les résultats clés des simulations lattice, figurent également à la fin de l’ouvrage.

\bigskip

%---------------------------------------------------------------------------------------
% Conclusion du Chapitre 1
%---------------------------------------------------------------------------------------
\noindent
\textbf{Conclusion du Chapitre 1 :} \\
Nous venons d’exposer le \emph{Mass Gap} en tant que défi majeur à la fois en physique théorique et en mathématiques modernes, et de souligner \textbf{l’objectif} : aboutir à une construction rigoureuse de Yang--Mills 4D exhibant un spectre massif. Dans le chapitre~\ref{chap:2}, nous dresserons un survol historique détaillé et rappellerons les motivations fondamentales (expérimentales, numériques, et purement théoriques) qui rendent cette entreprise si cruciale.

\vspace{2em}

%=======================================================================================
% Références Bibliographiques Utilisées dans ce Chapitre
% (En "dur" ici pour éviter toute omission)
%=======================================================================================
\begin{thebibliography}{99}
	
	\bibitem{Wilson1974}
	K.~G. Wilson,
	\textit{Confinement of Quarks},
	Phys.~Rev.~D \textbf{10}, 2445 (1974).
	\\[-0.75em]
	
	\bibitem{Creutz1980}
	M.~Creutz,
	\textit{Monte Carlo Study of Quantized SU(2) Gauge Theory},
	Phys.~Rev.~D \textbf{21}, 2308 (1980).
	\\[-0.75em]
	
	\bibitem{GrossWilczek1973}
	D.~J. Gross and F.~Wilczek,
	\textit{Ultraviolet Behavior of Non-Abelian Gauge Theories},
	Phys.~Rev.~Lett. \textbf{30}, 1343 (1973).
	\\[-0.75em]
	
	\bibitem{Politzer1973}
	H.~D. Politzer,
	\textit{Reliable Perturbative Results for Strong Interactions?},
	Phys.~Rev.~Lett. \textbf{30}, 1346 (1973).
	\\[-0.75em]
	
	\bibitem{Balaban1982-1}
	T.~Balaban,
	\textit{Renormalization Group Approach to Lattice Gauge Field Theories (I)},
	Commun.~Math.~Phys. \textbf{79}, 277--321 (1981).
	\\[-0.75em]
	
	\bibitem{Balaban1982-2}
	T.~Balaban,
	\textit{Renormalization Group Approach to Lattice Gauge Field Theories (II)},
	Commun.~Math.~Phys. \textbf{83}, 363--376 (1982).
	\\[-0.75em]
	
	\bibitem{Rivasseau1991}
	V.~Rivasseau,
	\textit{From Perturbative to Constructive Renormalization},
	Princeton University Press, Princeton (1991).
	
\end{thebibliography}

%=======================================================================================
% Fin du fichier : chap1.tex
%=======================================================================================
