%=======================================================================================
% Fichier : chap5.tex
% Chapitre 5 : La Formulation Multi-Échelle Constructive
%=======================================================================================
\chapter{La Formulation Multi-Échelle Constructive}
\label{chap:5}

%---------------------------------------------------------------------------------------
% Section 5.1 : Introduction aux Techniques de Renormalisation Constructive
%---------------------------------------------------------------------------------------
\section{Introduction aux Techniques de Renormalisation Constructive}
\label{sec:5.1}

En parallèle à la formulation sur réseau, la \textbf{renormalisation constructive} (ou \textit{multi-échelle}) s’appuie sur un \og découpage\fg\ en échelles de moment. Cette approche, développée par Balaban \cite{Balaban1982-1,Balaban1982-2}, Freedman, Rivasseau \cite{Rivasseau1991}, etc., vise à démontrer l’\textbf{existence} (au sens mathématique) de la mesure de Yang--Mills en 4D, \emph{directement} dans l’espace des champs continus.

\subsection*{Principe général}
\begin{itemize}
	\item \(\Lambda\) \textbf{UV} : on impose un cutoff en haute fréquence (ou haute impulsion) pour éliminer les divergences initiales.  
	\item \(\Lambda_{\mathrm{IR}}\) \textbf{finie} : on travaille d’abord dans un volume (ou un domaine en impulsion) fini, puis on fait tendre \(\Lambda_{\mathrm{IR}} \to 0\) ou le volume \(\to \infty\).  
	\item \textbf{Découpage en tranches} : on sépare l’intervalle des moments en \(\bigl[\Lambda_{\mathrm{IR}}, \Lambda\bigr]\) en morceaux \(\{\Lambda_{k+1},\Lambda_{k}\}\). Chaque \og bloc\fg\ est traité successivement via des intégrales fonctionnelles partielles.
\end{itemize}

\subsection*{Objectif}
Montrer qu’on peut, \textbf{à chaque étape de l’échelle}, absorber toutes les divergences potentielles par des \emph{contre-termes} de jauge \(\mathrm{SU}(N)\) (s’ils existent) et \textbf{conserver} la positivité (axiomes OS) ainsi que l’invariance de jauge.

\vspace{1em}

%---------------------------------------------------------------------------------------
% Section 5.2 : Découpage en Blocs d’Échelles (Approche Balaban, Rivasseau)
%---------------------------------------------------------------------------------------
\section{Découpage en Blocs d’Échelles (Approche Balaban, Rivasseau)}
\label{sec:5.2}

\subsection*{Schéma d’intégration par étapes}
On représente le champ de jauge \(A_\mu\) comme somme de composantes \(\phi_k\) de fréquences intermédiaires. L’intégrale de partition s’écrit en gros :
\[
\int \mathrm{d}\mu(A)\,\exp(-S[A]) \;=\;
\prod_{k=0}^{k_{\max}} \Bigl\{\int \mathrm{d}\mu_k(\phi_k)\Bigr\}\;\exp\bigl(-S_{\mathrm{eff},k}[\phi_k]\bigr),
\]
où \(\mathrm{d}\mu_k\) est la mesure gaussienne (ou quasi-gaussienne) à l’échelle \(k\). À chaque \emph{bloc}, on \textit{renormalise} pour éliminer les divergences accumulées.

\subsection*{Balaban’s lemmas}
T. Balaban \cite{Balaban1982-1,Balaban1982-2} a développé une série de lemmes techniques sur le \textbf{contrôle du flot} de renormalisation. Dans la formulation \(\mathrm{SU}(N)\), on a :
\begin{itemize}
	\item \textbf{Estimation} : la partie ultraviolette est gérable grâce à la liberté asymptotique ; pas de divergence insurmontable si on procède échelle par échelle.
	\item \textbf{Maintien de la jauge} : l’approche constructive respecte la structure non abélienne via la décomposition du champ \(A_\mu\) en blocs orthogonaux, tout en imposant une \og gauge fixing\fg\ partiel pour éviter les redondances.
\end{itemize}

\vspace{1em}

%---------------------------------------------------------------------------------------
% Section 5.3 : Théorie de l’Asymptotic Freedom et couplage à haute énergie
%---------------------------------------------------------------------------------------
\section{Théorie de l’Asymptotic Freedom et couplage à haute énergie}
\label{sec:5.3}

\subsection*{Rappels}
Comme vu au chapitre~\ref{chap:2} (section \ref{sec:2.3}), la \textbf{liberté asymptotique} implique que le couplage effectif \(g(\mu)\) devient \og petit\fg\ lorsque l’échelle de moment \(\mu\) est grande. En formulation multi-échelle :
\[
g^2(\Lambda_k) \;\sim\; \frac{1}{\ln(\Lambda_k/\Lambda_0)} \quad (\text{en 1-loop}).
\]

\subsection*{Importance pour la renormalisation constructive}
Puisque les tranches \(\{ \Lambda_{k+1}, \Lambda_k \}\) \emph{hautes énergies} (i.e. grands moments) sont traitées dans un régime \og faiblement couplé\fg, les expansions perturbatives autour d’un noyau gaussien restent contrôlées. Ce fait est crucial pour boucler la preuve qu’\textbf{aucune divergence} ne vient ruiner la construction en 4D \cite{Rivasseau1991,Freedman1982}.

\vspace{1em}

%---------------------------------------------------------------------------------------
% Section 5.4 : Renormalisation Ordre par Ordre : traitement des divergences UV
%---------------------------------------------------------------------------------------
\section{Renormalisation Ordre par Ordre : Traitement des Divergences UV}
\label{sec:5.4}

\subsection*{Idée perturbative initiale}
En théorie de champs, le \emph{renormalisation} se fait souvent \og ordre par ordre\fg\ en développant en séries de Feynman. Pour une théorie non abélienne en 4D, les contre-termes usuels sont de type \(\mathrm{Tr}(F_{\mu\nu}F^{\mu\nu})\), \(\mathrm{Tr}(A_\mu \partial_\nu A^\nu)\), etc.

\subsection*{Version constructive}
\begin{itemize}
	\item On ne s’appuie pas \textit{uniquement} sur la somme de graphes de Feynman, mais plutôt sur des \textbf{expansions en cluster}, en arborescences, etc.  
	\item Les \textbf{lemmes d’estimation} (Balaban, Freedman, Rivasseau) garantissent qu’à chaque étape d’intégration partielle, les nouvelles \og interactions\fg\ générées sont \emph{de même type} que celles déjà présentes (jauge invariante). On re-range ces interactions en \emph{counterterms} si besoin.
\end{itemize}

\subsection*{Résultat-clé}
En 4D, cette procédure \emph{ne diverge pas} tant que la \textbf{liberté asymptotique} tient son rôle en UV. Une fois franchie l’échelle \(\Lambda \to \infty\), la théorie renormalisée \(\mu_{\mathrm{YM}}\) se définit formellement comme une \emph{limite} cohérente de mesures partielles.

\vspace{1em}

%---------------------------------------------------------------------------------------
% Section 5.5 : Contrôle IR : volume fini, expansions en cluster
%---------------------------------------------------------------------------------------
\section{Contrôle IR : Volume Fini, Expansions en Cluster}
\label{sec:5.5}

\subsection*{Volume fini}
Pour éviter les \textbf{infra-rouges} (IR), on peut initialement travailler dans une boîte spatiale-temporelle de taille \(L^4\). Les bords peuvent être gérés par des conditions périodiques ou Dirichlet. L’objectif, comme dans la formulation sur réseau, est ensuite de prendre \(\,L\to\infty\).

\subsection*{Expansions en cluster}
\begin{itemize}
	\item \textbf{Principe} : on découpe l’espace en \og blocs\fg\ et on écrit la fonction de partition ou les corrélations comme une somme (ou exponentielle) d’intégrales \emph{factorisées}.  
	\item \textbf{Contrôle de l’exponentielle} : si le champ est \og massivement\fg\ confiné, la connectivité entre blocs lointains se réduit exponentiellement, suggérant une \emph{décroissance des corrélations}.
\end{itemize}

\subsection*{Conclusion sur l’approche multi-échelle}
Ainsi, la \textbf{construction de la mesure YM} en 4D s’appuie sur une suite d’étapes :  
\begin{enumerate}
	\item Découpage UV : \(\Lambda \to \infty\).  
	\item Contrôle IR : \(L \to \infty\).  
	\item Preuve que tous les axiomes OS sont satisfaits (voir chapitre~\ref{chap:8}).  
\end{enumerate}
Le \emph{mass gap} vient de l’exponentielle decay dans les corrélations, que nous aborderons plus en détail en partie IV.

\vspace{2em}

%---------------------------------------------------------------------------------------
% Conclusion du Chapitre 5
%---------------------------------------------------------------------------------------
\noindent
\textbf{Conclusion du Chapitre 5 :}\\
Les méthodes \textbf{multi-échelles constructives} permettent une \emph{démonstration rigoureuse} de l’existence de la théorie Yang--Mills 4D, en contrôlant pas à pas les divergences UV et IR. En pratique, c’est un \og puzzle\fg\ technique (lemmes de Balaban, expansions en cluster, etc.) qui converge vers la même conclusion que le \emph{lattice}: la théorie est \textbf{bien définie} dans la limite continuum.  
Au chapitre~\ref{chap:6}, nous mettrons en perspective ces deux approches (réseau et constructive), en discutant leur \textbf{équivalence formelle} et l’unicité de la théorie.

\vspace{2em}

%=======================================================================================
% Références Bibliographiques (en dur) pour Chapitre 5
%=======================================================================================
\begin{thebibliography}{99}
	
	\bibitem{Balaban1982-1}
	T.~Balaban,
	\textit{Renormalization Group Approach to Lattice Gauge Field Theories (I)},
	Commun.~Math.~Phys. \textbf{79}, 277--321 (1981).
	\\[-0.75em]
	
	\bibitem{Balaban1982-2}
	T.~Balaban,
	\textit{Renormalization Group Approach to Lattice Gauge Field Theories (II)},
	Commun.~Math.~Phys. \textbf{83}, 363--376 (1982).
	\\[-0.75em]
	
	\bibitem{Rivasseau1991}
	V.~Rivasseau,
	\textit{From Perturbative to Constructive Renormalization},
	Princeton University Press, Princeton (1991).
	\\[-0.75em]
	
	\bibitem{Freedman1982}
	D.~Freedman, K.~Johnson, J.~Latorre, 
	\textit{Differential Regularization and Renormalization: A New Method of Calculation in Quantum Field Theory},
	Nucl.~Phys.~B \textbf{371}, 353--414 (1992).
	
\end{thebibliography}

%=======================================================================================
% Fin du fichier : chap5.tex
%=======================================================================================
