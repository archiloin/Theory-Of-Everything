%=======================================================================================
% Fichier : chap6.tex
% Chapitre 6 : Comparaison et Unification des Deux Approches
%=======================================================================================
\chapter{Comparaison et Unification des Deux Approches}
\label{chap:6}

%---------------------------------------------------------------------------------------
% Section 6.1 : Équivalence Formelle des Limites (Réseau vs. Constructif)
%---------------------------------------------------------------------------------------
\section{Équivalence Formelle des Limites (Réseau vs. Constructif)}
\label{sec:6.1}

\subsection*{Réseau \textit{vs.} Continu multi-échelle}
On a présenté deux \textbf{régularisations} :
\begin{enumerate}
	\item \textbf{Lattice} : discretisation de l’espace-temps avec un pas \(a>0\).  
	\item \textbf{Multi-échelle} : impose un cutoff \(\Lambda\) en impulsion et reconstruit la mesure par blocs d’échelles successifs.
\end{enumerate}
Dans les deux cas, on prend des limites \(\,a \to 0\) ou \(\,\Lambda \to \infty\). L’\textbf{affirmation} est qu’on obtient \emph{la même théorie}, c.-à-d. la même collection de \emph{fonctions de corrélation} gauge-invariantes au continuum.

\subsection*{Points de vue mathématiques}
\begin{itemize}
	\item \textbf{Existence d’une unique mesure limite} : la cohérence du flot de renormalisation (Balaban et al.) garantit que la \og position\fg\ finale de la théorie dans l’espace des couplages est \textbf{indépendante} du chemin de régularisation (lattice ou cutoffs).  
	\item \textbf{Égalité des observables} : pour toute observable \(\mathcal{O}\) gauge-invariante, \(\langle \mathcal{O}\rangle_{\mathrm{lattice}}\) coïnciderait avec \(\langle \mathcal{O}\rangle_{\mathrm{constructif}}\) dans la limite continuum.
\end{itemize}

\vspace{1em}

%---------------------------------------------------------------------------------------
% Section 6.2 : Principes Généraux de Consistance
%---------------------------------------------------------------------------------------
\section{Principes Généraux de Consistance}
\label{sec:6.2}

\subsection*{Contrôle des divergences UV}
Dans les deux formulations, on contrôle \textbf{explicitement} les divergences UV :
\begin{itemize}
	\item Lattice : la somme sur un réseau fini (pas \(a\)) est \emph{forcément} finie ; on ne réintroduit l’infini qu’en faisant \(a\to 0\).  
	\item Constructif : on \emph{integre} échelle par échelle en maintenant un cutoff \(\Lambda\). Pas de diagramme de Feynman divergeant sans qu’un contre-terme (compatible jauge) intervienne.
\end{itemize}
Le succès tient en la \textbf{liberté asymptotique} en 4D, assurant qu’aucune singularité \og catastrophique\fg\ ne survient.

\subsection*{Invariance de jauge}
\begin{itemize}
	\item Sur le \textbf{lattice}, l’invariance de jauge est \emph{manifeste} (intégration de Haar indépendante).  
	\item En \textbf{constructif}, on veille à conserver la structure \(\mathrm{SU}(N)\) dans la décomposition multi-échelle, quitte à fixer partiellement la jauge pour supprimer les volumes de redondance, puis à vérifier que la dépendance résiduelle n’altère pas les observables physiques.  
\end{itemize}

\vspace{1em}

%---------------------------------------------------------------------------------------
% Section 6.3 : Discussion sur l’Unicité de la Théorie
%---------------------------------------------------------------------------------------
\section{Discussion sur l’Unicité de la Théorie}
\label{sec:6.3}

\subsection*{Une seule \og QCD\fg\ pure}
Par \textbf{unicité}, on entend qu’il n’y a pas, a priori, deux \og théories\fg\ distinctes de Yang--Mills 4D \(\mathrm{SU}(N)\) répondant aux mêmes axiomes. Les deux approches (lattice, multi-échelle) \emph{ne} donnent pas deux familles de solutions, mais au contraire une \textbf{même} mesure limite \(\mu_{\mathrm{YM}}\).

\subsection*{Cas de transitions de phase ?}
On pourrait craindre des transitions de phase (type \emph{Higgs} ou \emph{confinement/déconfinement}) qui mèneraient à plusieurs \og branches\fg\ non équivalentes. Toutefois, pour \(\mathrm{SU}(3)\) (et plus généralement \(\mathrm{SU}(N)\)) en 4D, la phase confinée demeure \textbf{unique} dans la limite continuum, sans seconde phase stable \cite{Greensite2003}.

\subsection*{Conclusion de la comparaison}
Ainsi, la \textbf{théorie de Yang--Mills 4D} est \textbf{unique} (pour un groupe de jauge \(\mathrm{SU}(N)\) donné) lorsqu’on respecte l’invariance de jauge et les axiomes QFT. Le réseau et le constructif ne sont que deux chemins \emph{différents} menant au même point final.

\vspace{2em}

%---------------------------------------------------------------------------------------
% Conclusion du Chapitre 6
%---------------------------------------------------------------------------------------
\noindent
\textbf{Conclusion du Chapitre 6 :}\\
Nous avons replacé côte à côte les deux \textbf{grandes approches} (régularisation sur réseau et renormalisation constructive) et souligné leurs points communs : chacune traite les divergences UV/IR de manière contrôlée, préserve la jauge, et aboutit à la \emph{même} théorie dans la limite continuum. Il existe donc \textbf{une seule} YM~4D (par groupe \(\mathrm{SU}(N)\)) répondant aux axiomes, ce qui renforce la cohérence globale du programme.  
Nous pouvons à présent entamer la \textbf{Partie III} (chapitres~\ref{chap:7} et \ref{chap:8}) pour détailler le \emph{passage à la limite continuum} et la satisfaction des axiomes d’Osterwalder--Schrader (puis la reconstruction Minkowski).

\vspace{2em}

%=======================================================================================
% Références Bibliographiques (en dur) pour Chapitre 6
%=======================================================================================
\begin{thebibliography}{99}
	
	\bibitem{Balaban1982-1}
	T.~Balaban,
	\textit{Renormalization Group Approach to Lattice Gauge Field Theories (I)},
	Commun.~Math.~Phys. \textbf{79}, 277--321 (1981).
	\\[-0.75em]
	
	\bibitem{Balaban1982-2}
	T.~Balaban,
	\textit{Renormalization Group Approach to Lattice Gauge Field Theories (II)},
	Commun.~Math.~Phys. \textbf{83}, 363--376 (1982).
	\\[-0.75em]
	
	\bibitem{Rivasseau1991}
	V.~Rivasseau,
	\textit{From Perturbative to Constructive Renormalization},
	Princeton University Press, Princeton (1991).
	\\[-0.75em]
	
	\bibitem{Greensite2003}
	J.~Greensite,
	\textit{The Confinement Problem in Lattice Gauge Theory},
	Prog.~Part.~Nucl.~Phys. \textbf{51}, 1--83 (2003).
	
\end{thebibliography}

%=======================================================================================
% Fin du fichier : chap6.tex
%=======================================================================================
