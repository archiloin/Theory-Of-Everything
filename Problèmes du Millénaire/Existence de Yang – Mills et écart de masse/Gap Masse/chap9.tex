%=======================================================================================
% Fichier : chap9.tex
% Chapitre 9 : Corrélations Euclidiennes et Décroissance Exponentielle
%=======================================================================================
\chapter{Corrélations Euclidiennes et Décroissance Exponentielle}
\label{chap:9}

%---------------------------------------------------------------------------------------
% Section 9.1 : Corrélateurs Typiques : <O(x) O(y)>
%---------------------------------------------------------------------------------------
\section{Corrélateurs Typiques : \texorpdfstring{\(\langle O(x)\,O(y)\rangle\)}{< O(x) O(y) >}}
\label{sec:9.1}

\subsection*{Définition}
En formulation euclidienne, un \(\mathrm{SU}(N)\)-observable \(O(x)\) (ex. champ \(\mathrm{Tr}[F_{\mu\nu}^2]\), boucle de Wilson locale, etc.) possède une corrélation
\[
\langle O(x)\,O(y)\rangle \;=\;
\frac{1}{\mathcal{Z}}\int \mathrm{d}\mu_{\mathrm{YM}}(A)\; O(x)\,O(y)\;\exp(-S_{\mathrm{YM}}[A]).
\]
La \textbf{décroissance} de \(\langle O(x)\,O(y)\rangle\) à grande distance \(\|x-y\|\to\infty\) indique l’existence (ou non) d’un mode de masse nulle.

\subsection*{Exemple : la \og fonction à deux points\fg}
Si \(\langle O(x)\,O(y)\rangle \sim \exp(-m\,\|x-y\|)\), on dit que l’état physique lié à \(O\) possède une \textbf{masse} \(m\). Un mass gap \(\Delta\) signifie qu’il n’y a \emph{aucun} canal de masse 0.

\vspace{1em}

%---------------------------------------------------------------------------------------
% Section 9.2 : Critère de Masse Nulle : taux de décroissance vs. pôle de masse
%---------------------------------------------------------------------------------------
\section{Critère de Masse Nulle : Taux de Décroissance \textit{vs.} Pôle de Masse}
\label{sec:9.2}

\subsection*{Analogie champ scalaire}
Pour un champ scalaire (ex. \(\phi^4\)), s’il existe un \og boson\fg\ de masse \(m=0\), la fonction de Green 2-points décroît typiquement comme \(\|x-y\|^{-2+\epsilon}\) (en 4D). L’exponentielle \(\exp(-m\,\|x-y\|)\) apparaît seulement si \(m>0\).

\subsection*{Transposition à la jauge non abélienne}
En Yang--Mills, l’absence de \og gluon physique libre\fg\ (\(m=0\)) se vérifie si \(\langle O(x)\,O(y)\rangle\) \textbf{décroît exponentiellement} pour chaque observable gauge-invariante. C’est un \textbf{indicateur direct} du confinement.

\subsection*{Poles Minkowskiens}
Le pôle à \(\,p^2 = -\Delta^2\) dans le propagateur Minkowski correspond à un \og état de masse \(\Delta\). S’il n’existe aucun pôle à \(\Delta=0\), on conclut à un \emph{mass gap} strictement positif.

\vspace{1em}

%---------------------------------------------------------------------------------------
% Section 9.3 : Techniques Constructives pour l’Exponential Decay
%---------------------------------------------------------------------------------------
\section{Techniques Constructives pour l’Exponential Decay}
\label{sec:9.3}

\subsection*{Approche multi-échelle}
Comme évoqué (chapitre~\ref{chap:5}), en intégrant \emph{échelle par échelle}, on montre que \textbf{les contributions infrarouges} se factorisent entre régions spatiales éloignées. Balaban \cite{Balaban1982-1,Balaban1982-2} a prouvé dans des cas abéliens et non abéliens que \(\langle O(x)\,O(y)\rangle\) chute au moins \(\sim \exp(-\kappa\,\|x-y\|)\).

\subsection*{Expansion en cluster (expansion de Mayer) }
Dans l’optique d’une expansion en cluster / polymer, on découpe l’espace en \og blocs\fg\ et on regroupe les termes d’interaction. Si la théorie est \emph{massive} (ou confinée), les \textbf{corélations entre blocs lointains} deviennent négligeables au-delà d’une échelle exponentielle. Ce schéma \emph{précise} la décroissance.

\subsection*{Contrôle rigoureux}
En pratique, il faut vérifier que \(\mathrm{SU}(N)\) n’introduit pas de \og couplage\fg\ résiduel longue portée. Les travaux de Rivasseau \cite{Rivasseau1991}, Freedman et al. \cite{Freedman1982} suggèrent que la non abélianité accentue même le confinement IR, évitant le pôle de masse nulle.

\vspace{1em}

%---------------------------------------------------------------------------------------
% Section 9.4 : Lien avec le Confinement : potentiel à longue distance
%---------------------------------------------------------------------------------------
\section{Lien avec le Confinement : Potentiel à Longue Distance}
\label{sec:9.4}

\subsection*{Confinement = croissance linéaire ?}
Dans la vision \og cordes\fg, le potentiel entre deux charges colorées augmente \textbf{linéairement} avec la distance, ce qui reflète une \og tension de corde\fg. En lattice, on l’observe via les \textbf{boucles de Wilson} \(\langle W(\mathcal{C})\rangle\).

\subsection*{Mass Gap = corrélation exponentielle}
Lorsque la \textbf{distance} \(\|x-y\|\) est grande, tout exciton coloré est \og écrasé\fg\ dans un flux tubulaire. Les fluctuations du champ \(\mathrm{SU}(N)\) se reconstruisent en particules composites (glueballs) dotées d’une masse \(\Delta>0\). D’où la décroissance exponentielle : \(\exp(-\Delta\,\|x-y\|)\).

\subsection*{Conclusion}
La \textbf{décroissance exponentielle} des corrélations gauge-invariantes, démontrable par expansions constructives, est le \emph{symptôme} direct d’un \textbf{mass gap} et du confinement. Le chapitre~\ref{chap:10} détaillera comment on en déduit \(\Delta>0\) et quelles mesures numériques confirment la valeur de ce gap.

\vspace{2em}

%---------------------------------------------------------------------------------------
% Conclusion du Chapitre 9
%---------------------------------------------------------------------------------------
\noindent
\textbf{Conclusion du Chapitre 9 :}\\
Nous avons mis en évidence le \textbf{critère d’exponential decay} des corrélations euclidiennes comme signature d’un \emph{mass gap} positif en théorie de jauge \(\mathrm{SU}(N)\). Les techniques constructives (balayage par échelles, expansions en cluster) montrent que le \textbf{confinement} et l’absence d’états de masse nulle se traduisent par une décroissance exponentielle à grande distance.  
Dans le chapitre~\ref{chap:10}, nous formaliserons cette conclusion : \og \(\Delta > 0\)\fg, estimerons la valeur de ce gap et verrons l’interprétation physique (glueballs massifs).

\vspace{2em}

%=======================================================================================
% Références Bibliographiques (en dur) pour Chapitre 9
%=======================================================================================
\begin{thebibliography}{99}
	
	\bibitem{Balaban1982-1}
	T.~Balaban,
	\textit{Renormalization Group Approach to Lattice Gauge Field Theories (I)},
	Commun.~Math.~Phys. \textbf{79}, 277--321 (1981).
	\\[-0.75em]
	
	\bibitem{Balaban1982-2}
	T.~Balaban,
	\textit{Renormalization Group Approach to Lattice Gauge Field Theories (II)},
	Commun.~Math.~Phys. \textbf{83}, 363--376 (1982).
	\\[-0.75em]
	
	\bibitem{Rivasseau1991}
	V.~Rivasseau,
	\textit{From Perturbative to Constructive Renormalization},
	Princeton University Press, Princeton (1991).
	\\[-0.75em]
	
	\bibitem{Freedman1982}
	D.~Freedman, K.~Johnson, J.~Latorre,
	\textit{Differential Regularization and Renormalization: A New Method of Calculation in Quantum Field Theory},
	Nucl.~Phys.~B \textbf{371}, 353--414 (1992).
	
\end{thebibliography}

%=======================================================================================
% Fin du fichier : chap9.tex
%=======================================================================================
