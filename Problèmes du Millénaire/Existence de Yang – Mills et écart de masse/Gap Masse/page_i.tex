%---------------------------------------------------------------
% Page i : Page de Titre
%---------------------------------------------------------------
\begin{titlepage}
	\thispagestyle{empty}
	\centering
	
	%-----------------------------------------------------------
	% Titre principal (créatif)
	%-----------------------------------------------------------
	{\Huge \bfseries
		Yang--Mills 4D :\\[0.2em]
		\textit{Au Cœur de la Cuisine Quantique} \\[0.2em]
		\large Démonstration "Pas à Pas" du Gap de Masse
		\par}
	
	\vspace{2cm}
	
	%-----------------------------------------------------------
	% Sous-titre (ou complément)
	%-----------------------------------------------------------
	{\Large
		\textbf{La Recette Complète :\\
			De la Lattice au Spectre Massif}
		\par}
	
	\vfill
	
	%-----------------------------------------------------------
	% Auteur / Date / Note sur le soutien
	%-----------------------------------------------------------
	{\large
		Auteur : \textsc{(AO)} \\[0.5em]
		Document rédigé le : \today
	}
	
	\vspace{1.5cm}
	
	% Mention "Aucun soutien financier"
	{\large \itshape
		Aucun soutien financier n'a été apporté pour la réalisation de ce travail.
	}
	
	\vfill
	
	%-----------------------------------------------------------
	% Résumé succinct (optionnel)
	%-----------------------------------------------------------
	\begin{minipage}{0.85\textwidth}
		\centering
		\textbf{Résumé :}\\[0.5em]
		Ce manuscrit présente une démonstration “pas à pas” de l'existence
		d’un \emph{gap de masse} strictement positif pour la théorie
		de Yang--Mills non abélienne en 4 dimensions. Nous montrons comment,
		à partir d'une régularisation sur réseau et/ou via des méthodes
		constructives multi-échelles, on aboutit à une théorie cohérente
		au sens des axiomes de champs quantiques (Osterwalder--Schrader).
		Enfin, nous prouvons la décroissance exponentielle des corrélations,
		garantissant un spectre discret avec un écart d'énergie (mass gap)
		strictement positif.
	\end{minipage}
	
	\vfill
	
\end{titlepage}
