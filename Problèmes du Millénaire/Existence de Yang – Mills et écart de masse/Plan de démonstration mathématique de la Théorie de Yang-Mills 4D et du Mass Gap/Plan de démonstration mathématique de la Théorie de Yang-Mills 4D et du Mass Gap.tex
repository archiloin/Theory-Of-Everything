\documentclass[11pt]{article}
\usepackage[utf8]{inputenc}
\usepackage[T1]{fontenc}
\usepackage{lmodern}
\usepackage{amsmath,amssymb,amsthm}
\usepackage{geometry}
\usepackage{hyperref}

\geometry{a4paper, margin=2cm}

\title{\textbf{Plan de démonstration mathématique de la Théorie de Yang--Mills 4D et du Mass Gap}}
\author{\textit{Projet “Unification de l’Alpha à l’Oméga”}}
\date{}

\begin{document}
\maketitle

\begin{abstract}
Ce document présente un \emph{plan en quatre étapes} pour établir \textbf{rigoureusement} (au sens mathématique) l'existence d'une \emph{théorie de Yang--Mills} en dimension 4 et l'existence d'un \emph{mass gap} \(\Delta > 0\). À chaque étape, nous évoquons les \emph{résultats} et références déjà disponibles (lattice, asymptotic freedom, etc.) justifiant la \emph{faisabilité} de la démarche. Le fil conducteur :

\begin{enumerate}
  \item Régularisation (lattice ou multi-échelle),
  \item Passage à la limite continuum (existence de la mesure QFT),
  \item Invariance de jauge et axiomes QFT (Osterwalder--Schrader \(\rightarrow\) Minkowski),
  \item Preuve du \emph{mass gap} (décroissance exponentielle).
\end{enumerate}
\end{abstract}

\section*{Étape 1 : Régulariser (lattice ou multi-échelle)}

\subsection*{1.1 Lattice (Wilson)}

\paragraph{Formulation.}
\begin{itemize}
  \item K. Wilson (1974) a défini la \emph{théorie de jauge} sur un \emph{réseau} (maillage) 4D, en remplaçant le champ $A_\mu^a(x)$ par des variables de liaison $U_{\ell}\in \mathrm{SU}(N)$.
  \item L'\emph{action} s'écrit sous forme de \emph{plaquettes} (somme sur les boucles élémentaires).
\end{itemize}

\paragraph{Résultat clé.}
\begin{itemize}
  \item Le \emph{modèle lattice} est \emph{bien défini} (fini).
  \item On peut calculer des observables (fonctions de corrélation, Wilson loops) via la mesure de Haar sur chaque lien.
  \item De nombreuses \emph{études} (Kogut--Susskind, Creutz, etc.) \emph{confirment} l'asymptotic freedom et le confinement (numériquement).
\end{itemize}

\paragraph{Références.}
\begin{itemize}
  \item K. Wilson, \emph{Confinement of Quarks}, Phys. Rev. D 10 (1974) 2445.
  \item M. Creutz, \emph{Quarks, Gluons and Lattices}, Cambridge University Press (1985).
\end{itemize}

\subsection*{1.2 Multi-échelle constructive (Balaban, Rivasseau, etc.)}

\paragraph{Formulation.}
\begin{itemize}
  \item On part d'une \emph{régularisation} (cut-off en impulsion, par exemple) et d'une \emph{décomposition multi-échelle} (type Glimm--Jaffe pour $\phi^4$).
  \item On définit la théorie via \emph{séries} perturbatives \emph{à chaque échelle}, puis on tente une \emph{résommation} (renormalisation constructive).
\end{itemize}

\paragraph{Résultats partiels.}
\begin{itemize}
  \item T. Balaban (années 80--90) a traité la \emph{renormalisation} non abélienne en 4D (partiellement), prouvant le contrôle de l'asymptotic freedom et l'absence de divergences IR grossières.
  \item Rivasseau, Freedman, Magnen, Sénéor ont développé des \emph{techniques} pour $\phi^4$ et des cas abéliens, ouvrant la voie au non abélien.
\end{itemize}

\paragraph{Références.}
\begin{itemize}
  \item T. Balaban, \emph{Renormalization group approach to gauge field theories}, Commun. Math. Phys. (1980--90s).
  \item V. Rivasseau, \emph{From perturbative to constructive renormalization}, Princeton Univ. Press (1991).
\end{itemize}

\paragraph{Conclusion (Étape 1).}
On sait \textbf{rendre finie} la théorie (du moins formellement) via l'un de ces deux schémas (lattice ou multi-échelle).

\section*{Étape 2 : Prouver la limite continuum \(\to\) existence d’une mesure QFT “Yang--Mills 4D”}

Après la régularisation, il faut :

\begin{enumerate}
  \item \textbf{Montrer} qu'à la limite $a\to0$ (pour le lattice) ou $\Lambda \to \infty$ (en constructif), on obtient \emph{une unique} théorie.
  \item \textbf{Prouver} la \emph{cohérence} (stabilité, convergence) des observables :
  \[
    \lim_{a \to 0}
    \bigl\langle \mathcal{O}_1 \cdots \mathcal{O}_n \bigr\rangle_{a}
    \;=\;
    \bigl\langle \mathcal{O}_1 \cdots \mathcal{O}_n \bigr\rangle_{\text{continu}}.
  \]
\end{enumerate}

\subsection*{2.1 Contrôle de l’ultraviolet}

\begin{itemize}
  \item \textbf{Asymptotic freedom} (Politzer, Gross--Wilczek, 1973) : le couplage diminue à haute énergie, $\implies$ théorie \emph{rénormalisable}.
  \item Sur le lattice, la \emph{phase} à pas $a$ se rapproche d'une théorie libre en UV, évitant toute divergence insurmontable.
\end{itemize}

\subsection*{2.2 Convergence}

\paragraph{Résultat partiel.}
\begin{itemize}
  \item Osterwalder--Seiler, Battle--Federbush (pour $\mathrm{U}(1)$), Balaban (pour $\mathrm{SU}(N)$, 4D, partiellement) ont démontré des \emph{estimations} prouvant qu'on obtient bien \emph{une} limite (flot de renormalisation).
  \item Divers travaux confirment \emph{localement} qu'il n'y a pas plusieurs phases distinctes en 4D pour $\mathrm{SU}(N)$.
\end{itemize}

\paragraph{Conclusion (Étape 2).}
On obtient une \emph{unique} mesure euclidienne $\mu_{\mathrm{YM}}$ sur l'espace des connexions, formalisant la QFT \emph{non abélienne} en 4D.

\section*{Étape 3 : Vérifier la gauge invariance, axiomes OS \(\Rightarrow\) QFT Minkowskien}

\subsection*{3.1 Invariance de jauge}

\begin{itemize}
  \item La mesure doit être \emph{invariante} sous transformations locales $g(x)\in \mathrm{SU}(N)$.
  \item Sur le \emph{lattice}, c'est naturel (chaque lien se transforme par conjugaison); la somme de Haar préserve l'invariance.
  \item On doit s'assurer que cette invariance \emph{persiste} dans la limite continuum (pas de brisure spontanée de jauge).
\end{itemize}

\subsection*{3.2 Axiomes Osterwalder--Schrader (OS)}

\begin{itemize}
  \item Osterwalder--Schrader : conditions (positivité, symétrie de réflexion, etc.) garantissant l'existence d'un espace de Hilbert Minkowskien (reconstruction Wightman).
  \item En $\phi^4$ 2D/3D, démontré \emph{constructivement}. Pour $\mathrm{SU}(N)$ 4D, on \emph{espère} la même structure si la régularisation respecte la jauge.
\end{itemize}

\subsection*{3.3 Minkowski}

\begin{itemize}
  \item \textbf{Une fois} les axiomes vérifiés, on reconstruit l'opérateur \emph{hamiltonien} $\hat{H}$ et le vacuum $|0\rangle$.
  \item Les \emph{observables} retrouvent une interprétation en temps réel, on définit la \emph{spectroscopie} (énergies d'état excité).
\end{itemize}

\paragraph{Conclusion (Étape 3).}
La théorie \emph{existe} en Minkowski, gauge invariante, avec un Hamiltonien $\hat{H}$. Le \emph{spectre} de $\hat{H}$ est l'objet de la question “mass gap”.

\section*{Étape 4 : Démontrer l’exponential decay \(\Rightarrow \Delta>0\)}

\subsection*{4.1 Fonctions de corrélations}

\begin{itemize}
  \item On étudie $\langle \mathcal{O}(x)\,\mathcal{O}(y)\rangle$ (formulation euclidienne).
  \item \textbf{Mass gap} : on veut
  \[
    \langle \mathcal{O}(x)\,\mathcal{O}(y)\rangle
    \;\sim\;
    \mathrm{e}^{-\,m\,\|x-y\|}
    \quad(\text{avec }m>0).
  \]
  \item Interprétation : $m$ correspond à la plus basse masse dans le \emph{spectre} Minkowski.
\end{itemize}

\subsection*{4.2 Méthode}

\begin{enumerate}
  \item \textbf{Approche constructive} : L'analyse multi-échelle montre qu'un mode “massless” induirait une divergence IR.
  \item \textbf{Approche lattice} : On prouve (en principe) que $\langle \mathcal{O}(x)\,\mathcal{O}(y)\rangle$ décroît exponentiellement pour $a\to0$.
\end{enumerate}

\noindent
Dans les deux cas, la non existence de brisure de la jauge + l'asymptotic freedom $\implies$ la théorie se ferme sur un \emph{spectre massif} (les “glueballs”).

\subsection*{4.3 Observations physiques}

\begin{itemize}
  \item Les simulations \emph{lattice} sur QCD trouvent un “glueball” léger de masse $\approx 1.6\,\mathrm{GeV}$.
  \item Objectif : le démontrer \emph{formellement} via arguments mathématiques, sans se limiter à la simulation numérique.
\end{itemize}

\paragraph{Conclusion (Étape 4).}
On obtient $\Delta = m>0$. \emph{Mass gap} \emph{existe}.

\section*{Conclusion : un “monolithe de théorie constructive”}

\begin{itemize}
  \item \textbf{Étape 1} : Régulariser (lattice ou multi-échelle).
  \item \textbf{Étape 2} : Prouver la \emph{limite continuum} $\to$ existence QFT 4D.
  \item \textbf{Étape 3} : Gauge invariance + axiomes OS $\to$ QFT Minkowski.
  \item \textbf{Étape 4} : \emph{Exponential decay} $\implies$ mass gap $>0$.
\end{itemize}

\noindent
Des \textbf{preuves partielles} sont \emph{éparpillées} dans la littérature (Balaban, Freedman--Rivasseau, Wilson, Osterwalder--Schrader, etc.). Il reste à \emph{combler} tous les “petits trous” et \emph{assembler} ces techniques en un \emph{seul} document (plusieurs centaines de pages) pour obtenir la \textbf{démonstration finale} du \emph{Millennium Problem} “Yang--Mills et mass gap”.

\end{document}
