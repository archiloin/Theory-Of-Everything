\documentclass[12pt]{article}
\usepackage[utf8]{inputenc}
\usepackage[T1]{fontenc}
\usepackage{lmodern}
\usepackage{amsmath,amssymb,amsfonts,amsthm}
\usepackage{csquotes}
\usepackage[french]{babel}
\usepackage{hyperref}
\pdfstringdefDisableCommands{%
  \def\Phi{Phi}%
  \def\Psi{Psi}%
  \def\kappa{kappa}%
  \def\gamma{gamma}%
  \def\delta{delta}%
  \def\mu{mu}%
  \def\nu{nu}%
  \def\lambda{lambda}%
  \def\alpha{alpha}%
  \def\beta{beta}%
  \def\leq{<=}%
  \def\geq{>=}%
  \def\int{\string\int}%
}
\sloppy
\overfullrule=5pt
\usepackage{graphicx}
\usepackage{geometry}
\geometry{margin=1in}

\title{\textbf{Un \og Th\'eor\`eme du Tout \fg{}~: De l'Unit\'e Primordiale \`a la Diversification des Forces}}
\author{Projet AIO (Alpha to Omega)}
\date{\today}

\begin{document}

\maketitle

\begin{abstract}
Nous pr\'esentons ici un \og Th\'eor\`eme du Tout \fg{} (ou \og Nous ne sommes qu'un \fg) 
qui propose une vision unifi\'ee des lois physiques \`a tr\`es haute \'energie. 
Nous d\'etaillons une action unifi\'ee \(\,U\) englobant la relativit\'e g\'en\'erale 
et les interactions de jauge, puis retra\c{c}ons la chronologie cosmique 
depuis l'\'echelle de Planck jusqu'aux conditions actuelles. 
L'\'evolution (refroidissement) induit la brisure de sym\'etrie \og GUT \fg{}, 
entraînant la \og d\'esunification \fg{} apparente en quatre forces fondamentales. 
Malgr\'e cette diversit\'e, nous plaidons qu'\`a l'origine, \og nous ne sommes qu'un \fg : 
toutes les interactions \& la mati\`ere d\'ecoulent d'un unique cadre coh\'erent. 
Ce sc\'enario, bien que partiellement sp\'eculatif, constitue l'une des visions les plus 
abouties d'une th\'eorie globale englobant \`a la fois la cosmologie primordiale 
et la physique moderne des particules.
\end{abstract}

\tableofcontents

\section{Introduction et motivations}
\label{sec:intro}

\subsection{Probl\'ematique}

Depuis le d\'ebut du XX\textsuperscript{e} si\`ecle, la physique est confront\'ee \`a un double succ\`es 
et \`a une double difficult\'e. D'une part, \textbf{la relativit\'e g\'en\'erale} \cite{einstein1915} 
d\'ecrit la gravit\'e comme g\'eom\'etrisation de l'espace-temps \`a l'\'echelle macroscopique, 
tandis que \textbf{la physique des particules} (mod\'elisation par des th\'eories quantiques de champs) 
expose trois interactions dites \og de jauge \fg : forte, faible et \'electromagn\'etique \cite{weinberg1995quantum, zee2010qft}.  
D'autre part, ces deux cadres peinent \`a s'assembler en une \textbf{th\'eorie unifi\'ee}, 
particuli\`erement \`a tr\`es haute \'energie (\textit{p.~ex.} \'echelle de Planck, 
\(\sim 10^{19}\,\mathrm{GeV}\)), o\`u la gravit\'e devrait \^etre quantifi\'ee.

\medskip

\noindent
\textbf{Questions-cl\'es} :
\begin{itemize}
    \item Comment concilier la \emph{Relativit\'e G\'en\'erale} et le \emph{Mod\`ele Standard} 
          (interactions forte, faible, \'electromagn\'etique) \`a tr\`es haute \'energie ?
    \item L'univers primordial \`a ces \emph{\'echelles extr\^emes} \(\sim 10^{15\text{-}19}\,\mathrm{GeV}\) 
          pr\'esente-t-il \emph{une sym\'etrie unifi\'ee} englobant toutes les forces ?
    \item Comment l'\emph{\'evolution cosmique} (expansion, refroidissement) a-t-elle 
          conduit \`a leur \og d\'esunification \fg{}, la formation de la mati\`ere ordinaire 
          et, plus tard, l'\'emergence de la vie ?
\end{itemize}

\subsection{Objectif}

Le pr\'esent expos\'e propose un \og Th\'eor\`eme du Tout \fg{} \emph{(Nous ne sommes qu'un)}, 
au sens d'une proposition centrale qui pose l'existence d'une unique \textbf{action} \(\,U\) : 
\[
U \quad\longmapsto\quad \delta U = 0 \quad \longmapsto\quad
\text{(\'equations de mouvement unifi\'ees)},
\]
int\'egrant \textbf{gravitation} (Einstein-Hilbert) et \textbf{interactions de jauge} (Yang-Mills), 
dont la \textbf{brisure de sym\'etrie} \`a plus basse \'energie s'exprime \`a travers 
un \textit{champ scalaire} de GUT (Grande Unification), puis \`a plus basse \'energie 
par le Higgs standard.

\subsection{Plan de l'expos\'e}

\begin{enumerate}
    \item[\S\ref{sec:theoreme}] \textbf{\'Enonc\'e du Th\'eor\`eme}~: \og Nous ne sommes qu'un \fg. 
          La physique \`a tr\`es haute \'energie forme un \emph{seul bloc} coh\'erent.
    \item[\S\ref{sec:actionU}] \textbf{L'\'equation unifi\'ee} \(\,U\)~: 
          expression de l'action englobant \`a la fois gravitation et jauge.
    \item[\S\ref{sec:chronologie}] \textbf{Chronologie de l'Univers}~: 
          depuis l'\'echelle de Planck jusqu'\`a l'\'epoque actuelle, 
          en passant par l'inflation, la grande unification, les transitions de phase.
    \item[\S\ref{sec:validation}] \textbf{Validation observationnelle}~: 
          indices cosmologiques, mesures exp\'erimentales (boson de Higgs, couplages de jauge, etc.).
    \item[\S\ref{sec:discussion}] \textbf{Discussion et conclusion}~: 
          les limites du sc\'enario, la question de la gravit\'e quantique et la perspective 
          d'\emph{un} univers r\'egi par une unique sym\'etrie initiale.
\end{enumerate}

\section{Enonc\'e du Th\'eor\`eme \og Nous ne sommes qu'un \fg}
\label{sec:theoreme}

\textbf{Th\'eor\`eme (Nous ne sommes qu'un).} 
\emph{Il existe une formulation unifi\'ee de la physique, repr\'esent\'ee par une unique action \(\,U\), 
o\`u la relativit\'e g\'en\'erale et les champs de jauge ne forment qu'un seul et m\^eme cadre \`a tr\`es haute \'energie. 
Les forces fondamentales (forte, faible, \'electromagn\'etique, gravit\'e) y apparaissent 
comme les diff\'erentes \og facettes \fg{} d'une sym\'etrie plus large, alors indiff\'erenci\'ee. 
L'expansion et le refroidissement de l'Univers induisent une s\'erie de \emph{brisures de sym\'etrie}, 
qui conduisent \`a la \og d\'esunification \fg{} et \`a la diversification des lois physiques. 
Malgr\'e cette multiplicité apparente \`a basse \'energie, toutes proviennent d'une m\^eme racine unique, 
d'o\`u l'id\'ee que \og nous ne sommes qu'un \fg.}

\begin{center}
\emph{Autrement dit, la s\'eparation des forces est une cons\'equence dynamique 
d'un \emph{champ unifi\'e primordial}, dont l'\'etat de sym\'etrie parfaite 
se \og casse \fg{} lors de la phase de refroidissement cosmique.}
\end{center}

\section{L'\'equation unifi\'ee \(\,U\)}
\label{sec:actionU}

Nous pr\'esentons d\'esormais l'action \(\,U\), dont la variation 
\(\,\delta U = 0\) donnera les \emph{\'equations de mouvement} 
pour la gravit\'e et les champs de jauge/mati\`ere \cite{weinberg1995quantum, georgi1974unified, langacker1981grand}.

\subsection{Expression canonique}

\begin{equation}
\label{eq:actionU}
\boxed{
\begin{aligned}
U
&=
\int d^4x \,\sqrt{-g}\;
\Bigl[\;
\frac{1}{2\kappa^2}\,R(g)
\;-\;\Lambda
\;-\;\tfrac14\,F_{\mu\nu}^A\,F^{\mu\nu A}
\notag \\
&\quad
+\;\overline{\Psi}\,\bigl(i\,\gamma^\mu D_\mu\bigr)\,\Psi
\;+\;|D_\mu \Phi|^2
\;-\;V(\Phi)
\;+\;\Delta_{\text{Yukawa}}
\;+\;\ldots
\Bigr]
\;+\;
S_{\text{corrections}}.
\end{aligned}
}
\end{equation}
\[
\text{o\`u}\quad
\kappa^2 = 8\pi G,\quad
F_{\mu\nu}^A = \partial_\mu A_\nu^A - \partial_\nu A_\mu^A + g\,f^{ABC}\,A_\mu^B\,A_\nu^C.
\]

\begin{itemize}
\item \(\tfrac{1}{2\kappa^2}\,R(g)\)~: terme d'Einstein-Hilbert (gravitation).  
\item \(\Lambda\)~: constante cosmologique (ou \'energie du vide).  
\item \(\,-\tfrac14\,F_{\mu\nu}^A\,F^{\mu\nu A}\)~: cin\'etique de jauge (interactions fortes, faibles, \'electromagn\'etiques) 
      via un groupe \(\mathcal{G}\) potentiellement unifi\'e (\(\mathrm{SU}(5), \mathrm{SO}(10)\), etc.).  
\item \(\,\overline{\Psi}\,(i \gamma^\mu D_\mu)\,\Psi\)~: repr\'esente les champs de fermions (quarks, leptons).  
\item \(\,|D_\mu\Phi|^2 - V(\Phi)\)~: champs scalaires, responsables de la \emph{brisure de sym\'etrie} GUT 
      puis plus tard \'electrofaible (\(\mathrm{SU}(2)_L\times \mathrm{U}(1)_Y \to \mathrm{U}(1)_{\mathrm{EM}}\)).  
\item \(\,\Delta_{\text{Yukawa}}\)~: couplages fermions--Higgs pour la g\'en\'eration de masses.  
\item \(\,S_{\text{corrections}}\)~: inclut des termes suppl\'ementaires (SUSY, termes topologiques, champs d'inflaton, \dots).
\end{itemize}

\paragraph{Lecture physique.}
\begin{itemize}
    \item \emph{Haute \'energie}~: on suppose que le groupe de jauge \(\mathcal{G}\) 
    est \textbf{non bris\'e}, la gravit\'e est potentiellement \textbf{quantique}.  
    \item \emph{Refroidissement}~: l'univers s'expanse, la temp\'erature chute, 
    un champ scalaire \(\,\Phi_\text{GUT}\) acquiert une valeur dans le vide 
    (brise \(\mathcal{G}\to G_1\times G_2 \times\dots\)), 
    puis \(\Phi_{\text{EW}}\) fait de m\^eme \`a plus basse \'energie, 
    brisant \(\mathrm{SU}(2)_L\times\mathrm{U}(1)_Y \to \mathrm{U}(1)_{\text{EM}}\).  
    \item \emph{Basse \'energie}~: on observe alors \(\mathrm{SU}(3)_\mathrm{c}\) (force forte), 
    \(\mathrm{SU}(2)_L\) et \(\mathrm{U}(1)_Y\) (for\-ces \'electrofaibles) 
    --- la gravit\'e est d\'ecoupl\'ee en pratique (effet extr\^emement faible).
\end{itemize}

\section{Chronologie de l'Univers : de l'Unit\'e \`a la D\'esunification}
\label{sec:chronologie}

Pour illustrer le \og Th\'eor\`eme \fg{} et l'action unifi\'ee \(\,U\), passons 
en revue les \textbf{grandes \'etapes} de l'Univers \cite{kolbturner, riess1998, planck2018}.

\subsection{\'Echelle de Planck (\(t \approx 10^{-43}\,\mathrm{s}\), \(E \sim 10^{19}\,\mathrm{GeV}\))}

\begin{itemize}
    \item La gravit\'e quantique devient in\'evitable: 
    fluctuations de l'espace-temps (\og mousse quantique \fg).  
    \item L'espace-temps peut \^etre courb\'e jusqu'\`a des dimensions de l'ordre 
          \(l_\mathrm{P} = \sqrt{\hbar\,G/c^3}\).  
    \item \emph{Possibilit\'e} qu'\`a cette \'echelle, la sym\'etrie soit \emph{encore plus large} 
          (ex. supergravité, M-th\'eorie, dimensions suppl\'ementaires).
\end{itemize}

\subsection{Phase d'inflation (hypoth\'etique, \(t \approx 10^{-36} \to 10^{-32}\,\mathrm{s}\))}

\begin{itemize}
    \item Expansion exponentielle : l'univers se \og gonfle \fg{}.  
    \item L'homog\'en\'eit\'e et l'isotropie sont accrues, tout en cr\'eant des fluctuations quantiques 
          \og gel\'ees \fg{} \`a grande \'echelle (germe de la structure future).  
    \item Sur le plan de l'action \eqref{eq:actionU}, on introduit souvent un \textbf{champ scalaire suppl\'ementaire} 
          (\og inflaton \fg), $S_{\text{corrections}}$ en tient compte, sans briser la cohérence du sc\'enario GUT.
\end{itemize}

\subsection{\'Echelle GUT (\(E \sim 10^{15\text{-}16}\,\mathrm{GeV}\))}

\begin{itemize}
    \item \textbf{Grande Unification} : 
    \(\mathrm{SU}(3)_\mathrm{c}, \mathrm{SU}(2)_\mathrm{L}, \mathrm{U}(1)_Y\) 
    se rassemblent dans un groupe plus grand (\(\mathrm{SU}(5)\), \(\mathrm{SO}(10)\), etc.).  
    \item Les quarks et leptons pourraient \^etre dans les m\^emes multiplets.  
    \item \textbf{Brisure GUT} : 
    un champ \(\,\Phi_{\mathrm{GUT}}\) (scalaire) acquiert une \emph{valeur dans le vide} 
    vers \(\sim 10^{15\text{-}16}\,\mathrm{GeV}\).  
    \item S\'epare d\'efinitivement la force forte de la force \'electrofaible.  
    \item Des bosons de jauge \og X \fg{} ou \og Y \fg{} massifs (non observ\'es encore) 
          pourraient m\'edier la \emph{d\'esint\'egration du proton}, 
          recherch\'ee via des exp\'eriences sp\'ecialis\'ees \cite{superk, kamland}.
\end{itemize}

\subsection{\'Echelle \'electrofaible (\(\approx 10^2\,\mathrm{GeV}\))}

\begin{itemize}
    \item \(\mathrm{SU}(2)_\mathrm{L}\times \mathrm{U}(1)_\mathrm{Y} \to \mathrm{U}(1)_\mathrm{EM}\).  
    \item Champ de Higgs du Mod\`ele Standard (\(\sim 125\,\mathrm{GeV}\)) : 
    brise la sym\'etrie \'electrofaible.  
    \item Bosons $W^\pm$, $Z^0$ massifs ; le \textbf{photon} reste sans masse.  
    \item Les fermions (quarks, leptons) acqui\`erent leurs masses 
          via les couplages de Yukawa \(\Delta_{\text{Yukawa}}\).
\end{itemize}

\subsection{Nucl\'eosynth\`ese primordiale (\(t \sim 1 \to 3\,\text{min}\), \(T \sim 1 \to 0.1\,\mathrm{MeV}\))}

\begin{itemize}
    \item Formation des noyaux l\'egers (H, He, Li) par la force forte et l'interaction faible 
          (contr\^olant le taux de conversion neutron--proton).  
    \item L'univers est alors domin\'e par les photons et la mati\`ere baryonique l\'eg\`ere.
\end{itemize}

\subsection{Recombinaison (\(t \sim 380\,000\,\mathrm{ans}, T \sim 0.3\,\mathrm{eV}\))}

\begin{itemize}
    \item Les \'electrons se lient aux noyaux, formant des atomes neutres.  
    \item Le \textbf{rayonnement} se d\'ecouple de la mati\`ere, constituant le \textbf{fond diffus cosmologique} (CMB).  
\end{itemize}

\subsection{Formation des structures et apparition de la vie}

\begin{itemize}
    \item Sous l'effet de la \textbf{gravitation}, la mati\`ere se rassemble en galaxies, en \'etoiles.  
    \item Les \'etoiles produisent par fusion thermonucl\'eaire des \'el\'ements lourds.  
    \item Sur des plan\`etes, la chimie (interaction \'electromagn\'etique) permet l'\'emergence 
          d'unit\'es auto-r\'epli\-catives (la vie).  
    \item Ainsi, \emph{toute la diversit\'e} (vivant, inerte, forces physiques) 
          d\'ecoule d'\emph{un m\^eme socle unifi\'e}.
\end{itemize}

\section{Indices observationnels et validations partielles}
\label{sec:validation}

\subsection{Brisure \'electrofaible confirm\'ee}

\begin{itemize}
    \item \textbf{Boson de Higgs} : d\'ecouvert en 2012 (ATLAS/CMS au LHC \cite{atlas2012higgs, cms2012higgs}), 
          validerait le m\'ecanisme de brisure \(\mathrm{SU}(2)_\mathrm{L}\times \mathrm{U}(1)_\mathrm{Y}\).  
    \item Mesures de pr\'ecision : masses des bosons $W, Z$, couplages de jauge coh\'erents 
          avec la brisure \'electrofaible.
\end{itemize}

\subsection{\'Evolution des constantes de couplage \og couplages running \fg}

\begin{itemize}
    \item Les mesures \`a haute \'energie montrent que 
          les constantes \(\alpha_3, \alpha_2, \alpha_1\) \emph{convergent} plus ou moins 
          aux alentours \(\sim 10^{15\text{-}16}\,\mathrm{GeV}\) \cite{amaldi1991precision}.  
    \item Sugg\`ere l'existence d'une \textbf{grande unification} \`a cette \'echelle.
\end{itemize}

\subsection{Pas de preuve directe de la GUT / gravit\'e quantique}

\begin{itemize}
    \item \textbf{Non observation} de la d\'esint\'egration du proton (pse~\(p \to e^+ \pi^0\)), 
    dont la dur\'ee de vie est sup\'erieure \`a \(10^{34}\,\mathrm{ans}\) selon les exp\'eriences.  
    \item \textbf{Gravit\'e quantique} : aucun test direct \`a \(\sim 10^{19}\,\mathrm{GeV}\).  
\end{itemize}

Malgr\'e cela, la \emph{coh\'erence interne} du sc\'enario, soutenue par des indices 
cosmologiques et particulaires, pointe vers une unification \`a plus haute \'energie.

\section{Discussion et conclusion}
\label{sec:discussion}

\subsection{Vers la compl\'etude du Th\'eor\`eme}

\begin{itemize}
    \item \emph{Action compl\`ete vs. sc\'enario effectif} : 
    l'action \eqref{eq:actionU} constitue un \og squelette \fg. 
    Les d\'etails exacts (choix du groupe GUT, introduction de la supersym\'etrie, 
    des invariants topologiques, du champ inflaton) varient d'un mod\`ele \`a l'autre.  
    \item \emph{Dimensions suppl\'ementaires} : 
    certains travaux (Kaluza-Klein, cordes) proposent une \textbf{unification g\'eom\'etrique} 
    dans un espace-temps plus grand.  
    \item \emph{R\^ole de l'inflation} : 
    on peut l'int\'egrer via un champ scalaire suppl\'ementaire dans $S_{\text{corrections}}$.  
\end{itemize}

\subsection{\og Nous ne sommes qu'un \fg : la r\'ealit\'e d'une racine commune}

Au niveau \textbf{philosophique}, cette id\'ee r\'ev\`ele que :
\begin{itemize}
    \item La \emph{mati\`ere vivante} et la \emph{mati\`ere inanim\'ee} s'appuient 
    sur les \emph{m\^emes briques} (fermions, bosons).  
    \item Les \emph{diff\'erentes forces} ne sont, \`a un certain niveau d'\'energie, 
    que \emph{diff\'erents aspects} d'une m\^eme interaction plus fondamentale.  
    \item L'\emph{\'evolution cosmique} \textbf{brise} la sym\'etrie originelle, 
    cr\'eant l'\og illusion \fg{} de la pluralit\'e alors qu'\`a la base, tout se tient.  
\end{itemize}

\paragraph{Conclusion~:} 
L'action unifi\'ee \eqref{eq:actionU} mod\'elise comment \emph{un seul cadre} 
--- \og nous ne sommes qu'un \fg --- se \og casse \fg{} en multiples lois. 
Le Th\'eor\`eme ainsi pos\'e sugg\`ere que la s\'eparation de la gravit\'e et 
des interactions de jauge n'est qu'une \emph{phase} d'\'energie \emph{basse}, 
tandis qu'\`a des \'energies extr\^emes, nous retournons \`a la racine unitaire.

\section{R\'ecapitulation du Th\'eor\`eme du Tout}

\paragraph{Synth\`ese.}
\begin{itemize}
    \item \textbf{Au commencement} (\(t \approx 0\)), l'univers se trouverait 
    sous le r\`egne d'une \emph{unique sym\'etrie} (ou d'un groupe GUT, 
    voire d'une sur-sym\'etrie plus vaste incluant la gravit\'e).  
    \item \(\,U\) (Eq.~\ref{eq:actionU}) est l'action \emph{unifi\'ee} 
    qui d\'ecrit \`a la fois l'Einstein-Hilbert (gravitation) 
    et le Yang-Mills (champs de jauge), ainsi que la mati\`ere (fermions, bosons de Higgs).  
    \item \textbf{Du fait de l'expansion et du refroidissement}, 
    un champ scalaire \(\,\Phi\) acquiert un v.e.v. \`a haute \'energie (\(\sim 10^{15\text{-}16}\,\mathrm{GeV}\))~: 
    \(\mathcal{G}\) se brise (GUT \(\to\) \(\mathrm{SU}(3)_\mathrm{c}\times\mathrm{SU}(2)_L\times\mathrm{U}(1)_Y\)).  
    \item \textbf{Plus tard} (\(\sim 10^2\,\mathrm{GeV}\)), la sym\'etrie \'electrofaible 
    se brise \`a son tour (\(\mathrm{SU}(2)_L\times \mathrm{U}(1)_Y \to \mathrm{U}(1)_\mathrm{EM}\)).  
    \item \textbf{\`A basse \'energie} (\(\ll 100\,\mathrm{GeV}\)), on observe 4 forces distinctes, 
    mais elles sont enracin\'ees dans la \emph{m\^eme action unifi\'ee}.  
\end{itemize}

\paragraph{Conclusion.}
Le \og Th\'eor\`eme du Tout \fg{} tient en cette phrase~: \emph{``Nous ne sommes qu'un''}. 
Toutes les entit\'es du monde physique (mati\`ere, lumi\`ere, gravit\'e, champs) 
proviennent d'un unique \emph{champ unifi\'e primordial}. 
Les diff\'erenciations et la complexit\'e du pr\'esent (de la structure galactique \`a la vie) 
r\'esultent de la \textbf{brisure s\'equentielle} de cette sym\'etrie initiale, 
au fil de l'histoire cosmique.

\bigskip

\begin{center}
\fbox{\parbox{0.88\textwidth}{
\centering
\textbf{«\,Nous ne sommes qu’un\,»~: tout r\'esulte d’une m\^eme structure unifi\'ee,}\\
\textbf{se diff\'erenciant au cours de l’histoire cosmique.}
}}
\end{center}

\section*{Remerciements et perspectives}

Ce sc\'enario, bien que partiellement \emph{th\'eorique}, offre une vision 
\textbf{int\'egr\'ee} entre la cosmologie primordiale et la physique des hautes \'energies. 
Il reste \`a le \textbf{tester} (d\'esint\'egration du proton, signatures directes 
de gravit\'e quantique, etc.) et \`a le \textbf{compl\'eter} (inflation, neutrinos, 
mati\`ere noire, \'energie sombre). 
Cependant, la \emph{coh\'erence interne} et les indices observationnels 
appuient l'id\'ee que l'unit\'e n'est pas un vain mot, mais bien 
un \textbf{principe directeur} de la nature.

\vspace{1em}

\begin{thebibliography}{99}

\bibitem{einstein1915}
A. Einstein,
\textit{Die Feldgleichungen der Gravitation (The Field Equations of Gravitation)},
Sitzungsberichte der Preussischen Akademie der Wissenschaften zu Berlin, 1915.

\bibitem{weinberg1995quantum} 
S. Weinberg, 
\textit{The Quantum Theory of Fields}, 
Cambridge University Press, 1995.

\bibitem{zee2010qft}
A. Zee,
\textit{Quantum Field Theory in a Nutshell},
Princeton University Press, 2010.

\bibitem{georgi1974unified}
H. Georgi \& S. L. Glashow,
``Unity of All Elementary-Particle Forces,''
\emph{Phys. Rev. Lett.}, 32, 438--441, 1974.

\bibitem{langacker1981grand}
P. Langacker,
``Grand Unified Theories and Proton Decay,''
\emph{Phys. Rept.}, 72, 185--385, 1981.

\bibitem{kolbturner}
E. W. Kolb \& M. S. Turner,
\textit{The Early Universe},
Addison-Wesley, 1990.

\bibitem{riess1998}
A. G. Riess et al.,
``Observational Evidence from Supernovae for an Accelerating Universe and a Cosmological Constant,''
\emph{The Astronomical Journal}, 116:1009--1038, 1998.

\bibitem{planck2018}
Planck Collaboration,
``Planck 2018 results. VI. Cosmological parameters,''
\emph{Astronomy \& Astrophysics}, \textbf{641}, A6, 2020.

\bibitem{superk}
Super-Kamiokande Collaboration,
\url{http://www-sk.icrr.u-tokyo.ac.jp/sk/}.

\bibitem{kamland}
KamLAND Collaboration,
\url{http://www.awa.tohoku.ac.jp/kamland/}.

\bibitem{atlas2012higgs}
ATLAS Collaboration,
``Observation of a new particle in the search for the Standard Model Higgs boson with the ATLAS detector at the LHC,''
\emph{Phys. Lett. B}, 716, 1--29, 2012.

\bibitem{cms2012higgs}
CMS Collaboration,
``Observation of a new boson at a mass of 125 GeV with the CMS experiment at the LHC,''
\emph{Phys. Lett. B}, 716, 30--61, 2012.

\bibitem{amaldi1991precision}
U. Amaldi, W. de Boer, \& H. F\"urstenau,
``Comparison of grand unified theories with electroweak and strong coupling constants measured at LEP,''
\emph{Phys. Lett. B}, 260(3-4), 447--455, 1991.

\end{thebibliography}

\end{document}
